%&thesis
\documentclass[a4paper,11pt,BCOR=8mm,twoside,headsepline]{scrbook}
\author{Daniel Suess}
\title{Due to, or in spite of? The effect of constraints on efficiency in quantum estimation problems}
%%%%%%%%%%%%%%%%%%%%%%%%%%%%%%%%%%%%%%%%%%%%%%%%%%%%%%%%%%%%%%%%%%%%%%%%%%%%%%%
%% Begin of dumped preamble
%%%%%%%%%%%%%%%%%%%%%%%%%%%%%%%%%%%%%%%%%%%%%%%%%%%%%%%%%%%%%%%%%%%%%%%%%%%%%%%
% \usepackage[T1]{fontenc}
\usepackage{ifpdf}
\ifpdf{}
  \usepackage[utf8]{inputenc}
\fi
\usepackage{ifluatex}
\ifluatex{}
  \usepackage[utf8]{luainputenc}
  \usepackage{shellesc}
\fi

\usepackage{graphicx,subcaption}
\usepackage[backend=biber,style=alphabetic,bibencoding=utf8]{biblatex}
\bibliography{references_ds}

\usepackage[color=red!50!white,textsize=small,textwidth=2.5cm]{todonotes}

% Externalize tikz pictures except for todonotes
\usepackage{tikz}%for the figures
\usetikzlibrary{external,calc,backgrounds}
\tikzexternalize[prefix=build/]
\tikzset{external/mode=graphics if exists}
% \tikzset{external/system call={lualatex \tikzexternalcheckshellescape -halt-on-error -interaction=batchmode -jobname "\image" "\texsource"}}


\makeatletter
\renewcommand{\todo}[2][]{\tikzexternaldisable\@todo[noline,#1]{#2}\tikzexternalenable}
\makeatother
\usepackage{enumitem}


%%%%%%%%%%%%%%%%%%%%%%%%%%%%%%%%%%%%%%%%%%%%%%%%%%%%%%%%%%%%%%
\usepackage[breaklinks=true]{hyperref}
%getting rid of hyperref's ugly boxes.
%From:http://tex.stackexchange.com/a/51349
\hypersetup{
  colorlinks   = true, %Colours links instead of ugly boxes
  urlcolor     = blue, %Colour for external hyperlinks
  linkcolor    = blue, %Colour of internal links
  citecolor   = red %Colour of citations
}

% if \makefull is defined (e.g. by commandline argument, include everything)
\ifdefined\makefull\else
  \includeonly{chapters/introduction,chapters/conclusion}
  % \includeonly{chapters/error_region,chapters/error_region_appendix}
  % \includeonly{chapters/phaselift,chapters/phaselift_appendix}
  % \includeonly{chapters/tensors,chapters/tensors_appendix}
\fi

%%%%%%%%%%%%%%%%%%%%%%%%%%%%%%%%%%%%%%%%%%%%%%%%%%%%%%%%%%%%%%%%%%%%%%%
%                             Text Macros                             %
%%%%%%%%%%%%%%%%%%%%%%%%%%%%%%%%%%%%%%%%%%%%%%%%%%%%%%%%%%%%%%%%%%%%%%%

\newcommand{\quotes}[1]{{``#1''}}

%%%%%%%%%%%%%%%%%%%%%%%%%%%%%%%%%%%%%%%%%%%%%%%%%%%%%%%%%%%%%%%%%%%%%%%
%                             Math Macros                             %
%%%%%%%%%%%%%%%%%%%%%%%%%%%%%%%%%%%%%%%%%%%%%%%%%%%%%%%%%%%%%%%%%%%%%%%
\newcommand{\Prob}{\mathbb{P}}
\newcommand{\Exp}{\mathbb{E}}

\newcommand{\abs}[1]{{\vert#1\vert}}
\newcommand{\Abs}[1]{{\left\vert#1\right\vert}}
\newcommand{\norm}[1]{{\Vert#1\Vert}}
\newcommand{\Norm}[1]{{\left\Vert#1\right\Vert}}

\makeatletter
\@ifpackageloaded{hyperref}{}{\usepackage{hyperref}}
\makeatother

\usepackage{minted}

\newenvironment{pythoncode}{%
  \VerbatimEnvironment\begin{minted}{python}%
  }{%
  \end{minted}%
}
\newenvironment{pythonoutput}{%
  \VerbatimEnvironment\begin{minted}{python}%
  }{%
  \end{minted}%
}


%% use \[ ... \] for displayed equations
\newcommand{\myequation}{\begin{equation}}
\newcommand{\myendequation}{\end{equation}}
\let\[\myequation{}
\let\]\myendequation{}

%% Only referenced equations are numbered
% \usepackage{autonum}

%%%%%%%%%%%%%%%%%%%%%%%%%%%%%%%%%%%%%%%%%%%%%%%%%%%%%%%%%%%%%%
%%%%%%%%%%%%%%%%%%%%%%%%%%%%%%%%%%%%%%%%%%%%%%%%%%%%%%%%%%%%%%%%%%%%%%%%%%%%%%%
\begin{document}

\frontmatter
\maketitle
%%%%%%%%%%%%%%%%%%%%%%%%%%%%%%%%%%%%%%%%%%%%%%%%%%%%%%%%%%%%%%%%%%%%%%%%%%%%%%%
\chapter*{Abstract}

In this thesis, we study the interplay of constraints and complexity in quantum estimation.
We investigate three inference problems, where additional structure in the form of constraints is exploited to reduce the sample and/or computational complexity.
The first example is concerned with uncertainty quantification in quantum state estimation, where the positive-semidefinite constraint is used to construct more powerful, that is smaller, error regions.
However, as we show in this work, doing so in an optimal way constitutes a computationally hard problem, and therefore, is intractable for larger systems.
This is in stark contrast to the unconstrained version of the problem under consideration.
The second inference problem deals with phase retrieval in general, and characterizing linear optical circuits in particular.
The main challenge here is the fact that the measurements are oblivious to phases, and hence, their reconstruction requires deliberate interference.
We propose a reconstruction algorithm based on ideas from low-rank matrix recovery, which exploits an exact rank-one constraint on the signal to be  recovered.
Furthermore, in this work we propose a measurement ensemble tailored to the specific application of characterizing linear optical devices.
Finally, we investigate low-rank tensor recovery -- the problem of reconstructing a low-complexity tensor embedded in an exponentially large space.
We provide evidence that a low-rank tensor can be recovered from a number of measurements that only depends on its intrinsic complexity, and hence, scales polynomially in the order of tensor.
Therefore, the low-rank constraint can be exploited to dramatically reduce the sample complexity of the problem.
By using efficiently representable measurement tensors, our approach is also computationally efficient.

%%%%%%%%%%%%%%%%%%%%%%%%%%%%%%%%%%%%%%%%%%%%%%%%%%%%%%%%%%%%%%%%%%%%%%%%%%%%%%%
\chapter*{Kurzzusammenfassung}

\todo[inline]{TODO}

%%%%%%%%%%%%%%%%%%%%%%%%%%%%%%%%%%%%%%%%%%%%%%%%%%%%%%%%%%%%%%%%%%%%%%%%%%%%%%%
\tableofcontents
\newpage
\makeatletter
\providecommand\@dotsep{5}
\makeatother
\listoftodos\relax
\newpage

%%%%%%%%%%%%%%%%%%%%%%%%%%%%%%%%%%%%%%%%%%%%%%%%%%%%%%%%%%%%%%%%%%%%%%%%%%%%%%%
\mainmatter

 % -*- root: ../thesis.tex -*-

\chapter{Introduction}%
\label{chap:introduction}


% physics -> empirical science -> knowledge only thrrough observation/experiments
% quantitative outcomes can fluctuate from one meas. to another due to imperfections (uncontrollable degrees of freedom in the system, measurement errors, etc.)
% additionally, measurement outcomes in quantum mechanics inherently random. 
% necessitates



% 

 % -*- root: ../thesis.tex -*-
\chapter{Uncertainty Quantification for quantum state estimation}
\label{chap:error}


%%%%%%%%%%%%%%%%%%%%%%%%%%%%%%%%%%%%%%%%%%%%%%%%%%%%%%%%%%%%%%%%%%%%%%%%%%%%%%%%
\section{Introduction to Statistics}
\label{sec:error.intro}

% two main flavours of statistics...
% task of parameter estimation, model
% parametric model: state space \Omega
However, even if our model describes the data perfectly we cannot exactly recover this value from a finite amount of data due to statistical fluctuations.
The concept of error bars, or more generally error regions, allows for quantifying the uncertainty of a given estimate. 

% uncertatiny quantification, problem with point estimators

\subsection{Frequentist Statistics}
\label{sub:intro.frequentist}

In the frequentist (or orthodox) framework, the probability of an outcome of a random experiment is defined in terms of its relative frequency of occurrence when the number of repetitions goes to infinity~\cite{Keynes_2007_Treatise,Kiefer_2012_Introduction}.
More precisely, denote the number of repetitions of an experiment by $T$ and the number of times the event under consideration $x$ occurred by $n_T$. 
Then, a Frequentist interprets the probability $\Prob(x)$ as the statement that if $T \to \infty$, $\frac{n_T}{T} \to \Prob(x)$.

% probabliities = hypothetical frequencies, not repeatable
For the task of parameter estimation, we assume that the observed data are generated from the parametric model with \quotes{true} parameter $\theta \in \Omega$, which is unknown.
From a finite number of observations $X_1, \ldots X_N$, we must construct an estimate for $\theta$ that is close to the true value in some sense.
The function $\hat\theta$ that maps observations to such an estimate is called a point estimator.
The quality of an estimator is measured by its risk function.
A risk function commonly used for continuous parameter spaces is the mean square error
\[
  \label{eq:frequentist.mean_square}
  \mathcal{L}_{\hat\theta}(\theta) := \Exp_\theta \left( \Norm{\theta - \hat\theta(X_1, \ldots, X_N)}^2 \right).
\]
Note that \cref{eq:frequentist.mean_square} -- and therefore the performance of a given estimator -- still depends on the unknown true value $\theta$.
Strategies to make statements independent of the true value include 
\begin{definition}
  \label{def:frequentist.optimality_conditions}
  \begin{itemize}
    \item $\hat\theta$ is called a \emph{uniformly best} estimator, if for all other estimators $\hat\theta'$ and all values of the true parameter $\theta \in \Omega$
    \[
      \mathcal{L}_{\hat\theta}(\theta) \le \mathcal{L}_{\hat\theta'}(\theta).
    \]

    \item $\hat\theta$ is called \emph{minimax}, if for all other estimators $\hat\theta'$
    \[
      \sup_{\theta\in\Omega} \mathcal{L}_{\hat\theta}(\theta) \le \sup_{\theta\in\Omega} \mathcal{L}_{\hat\theta'}(\theta).
    \]

    \item $\hat\theta$ is called \emph{best on average} w.r.t.\ a distribution of the true value $\theta \sim \Theta$ if for all other estimators $\hat\theta'$
    \[
      \Exp_{\theta \sim \Theta} \mathcal{L}_{\hat\theta}(\theta) \le  \Exp_{\theta \sim \Theta} \mathcal{L}_{\hat\theta'}(\theta).
    \]

    \item $\hat\theta$ is called \emph{admissible} if there is no other estimator $\hat\theta'$ such that
    \[
      \forall\theta \in \Omega\colon \mathcal{L}_{\hat\theta'}(\theta) \le \mathcal{L}_{\hat\theta}(\theta) 
      \quad\mbox{ and }\quad
      \exists\theta \in \Omega\colon \mathcal{L}_{\hat\theta'}(\theta) < \mathcal{L}_{\hat\theta}(\theta) 
    \]
  \end{itemize}
\end{definition}
\todo{Citation}
\todo{Discussion of different properties}
\todo{Choice is arbitrary!}
\todo{How does this connect with definition of probability?}


% examples (also depends on notion of volume), admissability
However, as already mentioned in the introduction, point estimators cannot convey uncertainty in the estimate. 
For this purpose we need to introduce a precise notion of \quotes{error bars}, namely \emph{confidence regions}.
A confidence region $\CR \subset \Omega$ with coverage $\alpha \in [0,1]$ is a region estimator -- that is a function that maps observed data to a subset of the parameter space -- such that the true parameter is contained in $\CR$ with probability greater than $\alpha$
\[
  \label{eq:frequentist.coverage}
  \forall \theta\in\Omega \colon \Prob_\theta\left( \CR(X_1, \ldots, X_N) \ni \theta \right) \ge \alpha.
\]
Similar to point estimators, \cref{eq:frequentist.coverage} does not uniquely determine a confidence region construction.
Furthermore, additional constraints are necessary to exclude trivial constructions such as the following:
Take the region estimator, which is always equal to the full parameter space independent of the data $\CR(X_1, \ldots X_N) = \Omega$, then 
\[
  \Prob_\theta\left(  \CR(X_1, \ldots, X_N) \ni \theta  \right) = 1 \ge \alpha
\]
for all confidence levels $\alpha$.
Although, this construction trivially fulfils the coverage condition~\eqref{eq:frequentist.coverage}, it does not provide useful information on the uncertainty as it does not restrict the parameter space at all.
Therefore, we have to impose a notion of what constitutes a good confidence region.

Clearly, if we have two confidence regions $\CR_1$ and $\CR_2$ with the same confidence level $\alpha$ and $\CR_1 \subset \CR_2$, then $\CR_1$ is more informative.
More generally, smaller regions should be preferred since they convey more confidence in the estimate and exclude more alternatives.
Therefore, measures of size such as (expected) volume or diameter are commonly used as risk functions for region estimators.
This leads to similar definitions as in \cref{def:frequentist.optimality_conditions} for confidence regions with the additional constraint that \cref{eq:frequentist.coverage} is fulfilled.

\todo{Today: Asymptotic optimal, adaptive, ...}
 % -*- root: ../thesis.tex -*-
\chapter{Phaselift}
\label{chap:phaselift}


 % -*- root: ../thesis.tex -*-
\chapter{Tensors}
\label{chap:tensors}

In the last chapter, we discussed a solution to the phase retrieval problem based on ideas from \emph{low-rank matrix recovery}.
The latter can be summarized as follows.
Consider a matrix\footnote{%
  Here, we choose $X$ to be real to simplify some notation -- an adaption of the statements made to the complex case is straight forward.
}
$X \in \Reals^{d_1 \times d_2}$ of rank $r$ that is $X$ has only $r$ non-zero singular values
The goal is to reconstruct $X$ from $m$ linear measurements of the form
\[
  y_l = {\braket{A\ind{l}, X}}, \quad l=1,\ldots,m
  \label{eq:tensors.lin_measurements}
\]
where $\braket{A, B} = \tr (\transpose{A} B)$ denotes the Frobenius inner product, $A\ind{l} \in \Reals^{d_1 \times d_2}$ the \emph{measurement matrices}, and $y_l \in \Reals$ the corresponding measurement data.
In other words, we are looking for a solution $X$ of the linear problem defined by \cref{eq:tensors.lin_measurements}.
In case the $A_i$ span $\Reals^{d_1 \times d_2}$, this can be solved simply by computing the (pseudo-)inverse similar to \cref{sub:ortho.linear_inversion}.
However, in the underdetermined case $m < d_1 d_2$, this constitutes an ill-posed linear problem.
Nevertheless, we can still single out a unique solution in some cases by exploiting the additional structure, namely the low-rank assumption.
\todo[noline]{Check log factors}
For example, in \cref{thm:FIXME} we show that $m = \Order(d)$ random measurements sampled from an appropriate distribution are sufficient to reconstruct any positive semidefinite, Hermitian $X \in \Complex^{d \times d}$ of rank one.
More generally, we can recover any $X \in \Reals^{d_1 \times d_2}$ with $\rank X = r$ from $m = \Order(d_1 + d_n) r$ randomly chosen measurements~\cite{FIXME}.
Note that this is asymptotically optimal since we need at least $(d_1 + d_2) r$ real parameters to specify the left- and right-singular vectors of $X$.
For more details see~\cite{Li_Paper}
But \cref{thm:FIXME} not only guarantees identifiability of $X$, it also provides a positive semidefinite program to compute the reconstruction.
The existence of such an efficient algorithm is crucial since the straight-forward reconstruction via rank minimization
\[
  \estim{X} = \argmin_{X'} \rank X' \quad\mathrm{s.t.}\ \braket{A\ind{i}, X'} = y_i (i = 1, \ldots, m)
\]
is $\NP$-hard~\cite{FIXME}.

A closely related problem is studied under the name of \emph{compressed sensing}:
The goal is to recover an $s$-sparse vector $X \in \Reals^d$ from linear measurements as in \cref{eq:tensors.lin_measurements} with $\braket{\cdot, \cdot}$ denoting the Euclidean scalar product on $\Reals^d$.
Here, a vector is $s$-sparse if it has $s$ non-vanishing components with respect to a certain basis.
Therefore, the property of having low-rank may be regarded as a non-commutative analogue of sparsity since a matrix has low-rank if and only if it is sparse in its eigenbasis.
\todo[noline]{Check log factors}
Similar to the matrix case, $X$ can be recovered from $m = \Order(s \log d)$ randomly chosen measurement vectors $A\ind{l} \in \Reals^d$ using a linear program~\cite{Rauhut}.
Hence, the number of measurements required scales with the \emph{intrinsic complexity} of the sparse signal vector:
If we encode each component of $X$ using $N$ bits, we can encode the full vector using
\[
  N \times s + s \log d = \Order(s \log d)
\]
bits.
The second summand on the left hand side is necessary to encode the positions of the non-zero components.

In conclusion, both compressed sensing and low-rank matrix recovery provide techniques to efficiently reconstruct a low-complexity, compressible signal embedded in a higher-dimensional space from linear measurements.
Here, \quotes{efficiently} refers to both, low computational as well as sample complexity -- that is the number of measurements required scales with the intrinsic complexity of the signal and not the dimension of the ambient space.\\



In this chapter, we consider a natural extension of the ideas introduced above to the problem of recovering low-rank tensors.
For this purpose, consider a tensor $X \in \Reals^{d_1 \times \cdots \times d_N}$.
If not stated otherwise, we assume $d_1 = \cdots = d_N = d$ throughout this chapter.
Contrary to the case of matrices, there are many inequivalent definitions of \quotes{rank} for tensors.
The most natural generalization of the matrix rank is given by the \emph{canonical tensor rank} of $X$, which is defined by the smallest number of rank-1 tensors that add up to $X$~\cite{Kolda}
In other words, the canonical tensor rank is the smallest number $r$ such that
\[
  X = \sum_{l=1}^r a\ind{l}_1 \otimes \cdots a\ind{l}_N
  \label{eq:tensors.canonical_decomposition}
\]
with $a\ind{l}_i \in \Reals^{d_i}$.
The decomposition~\eqref{eq:tensors.canoncial_decomposition} is known under many names such as \emph{CANDECOMP/PARAFAC}~\cite{Kolda}.
Although it captures the natural structure of many problems well~\cite{FIXME} and only requires $\Order(r N d)$ parameters, it has shortcomings in practice:
Not only does the best rank-$r$ approximation not exist for certain tensors~\cite{Kolda}, it is also hard to compute if it does~\cite{FIMXE}.
In general, the CANDECOMP/PARAFAC decomposition is often hard to deal with computationally.
\todo{Some compressed sensing results for canonical low-rank tensors}

A different notion of tensor rank is induced by the tensor decomposition known under many different names such as \emph{matrix-product states} (MPS) and \emph{finitely correlated states} in quantum physics or \emph{tensor train} in the applied math community.
We introduce the exact definition in \cref{sec:tensors.mps}, but let use point out that a tensor $X \in \Reals^{d^{\otimes N}}$ with MPS-rank $r$ can be parametrized by $\Order{r^2 N d}$ real numbers.
This motivated the main question we are trying to answer in this chapter:
Can we efficiently reconstruct a tensor with low MPS-rank from few linear measurements?
In the ideal case, the number of measurements required for reconstructing a tensor with low MPS-rank would scale as $m = \Order{r^2 N d}$, which is an exponential improvement over the naïve linear inversion requiring $m = d^N$.
In contrast, compressed sensing or low-rank matrix recovery techniques can only yield polynomial improvement in the sample complexity.

This chapter is structured as follows:
In \cref{sec:tensors.mps}, we introduce the MPS tensor decomposition.
\Cref{sec:tensors.als} reports on work in progress, which is concerned with recovering tensors with low MPS-rank from few linear measurements using an \emph{Alternating Least Squares} (ALS) algorithm.
Finally, in \cref{sec:tensors.mpnum} we present the software library \mpnum dealing with tensors in MPS representation.
\mpnum was developed as part of this work to facilitate numerical computations in a user friendly and reusable manner.


\subsection*{Relevant publications}
\begin{itemize}
  \item Ž.\ Stojanac, D.\ Suess, M.\ Kliesch: \textit{On the distribution of a product of N Gaussian random variables}, Proceedings Volume 10394, Wavelets and Sparsity XVII; 1039419 (2017)
  \item Ž.\ Stojanac, D.\ Suess, M.\ Kliesch, \textit{On products of Gaussian random variables}, arXiv:1711.10516
  \item D.\ Suess, M.\ Holzaepfel, \textit{mpnum: A matrix product representation library for Python}, ???
\end{itemize}
%%%%%%%%%%%%%%%%%%%%%%%%%%%%%%%%%%%%%%%%%%%%%%%%%%%%%%%%%%%%%%%%%%%%%%%%%%%%%%%%
\section{Matrix Product States}%
\label{sec:tensors.mps}

% graphical notation
% connected to the MPS/TT tensor decomposition; first use: Hidden Markov model -> factorization of probabliilty distributions. Quantum Physics -> Werner (finetly correlated states); tensor train -> machine learning
%   - here: MPS/TT rank: numerical advantageous, captures structure well of many problems in physics; especially well suited for linear local structure; recently quite popular
% low-rank tensor = low TT rank tensor

%%%%%%%%%%%%%%%%%%%%%%%%%%%%%%%%%%%%%%%%%%%%%%%%%%%%%%%%%%%%%%%%%%%%%%%%%%%%%%%%
\section{ALS for Tensors}%
\label{sec:tensors.als}

% big problem: standard model (fully Gaussian measurements) also not efficient; partially solved in Zeljka

\subsection{Theory}%
\subsection{Numerics}%
% Implementation details; how to speed up initialization, why suitable for GPU%

%%%%%%%%%%%%%%%%%%%%%%%%%%%%%%%%%%%%%%%%%%%%%%%%%%%%%%%%%%%%%%%%%%%%%%%%%%%%%%%%
\section{The Python Library mpnum}%
\label{sec:tensors.mpnum}


 % -*- root: ../thesis.tex -*-

\chapter{Conclusion}%
\label{chap:conclusion}

To summarize, we have investigated three inference problem from quantum physics, which are subject to different types of constraints.
The main motivation of this work stems from the observation that exploiting additional structure in inference problems often helps to reduce their sample complexity.
However, the examples in this work differ in how taking the constraints into account affects their computational complexity.\\



The first inference problem under consideration is quantum state estimation -- reconstructing the density matrix of a quantum system from measurements.
More specifically, we are interested in optimal error regions that take the positive semi-definite constraint of physical states into account.
For this purpose, we show that deciding whether an ellipsoid is contained in the set of positive semi-definite matrices is $\NP$-hard.
As a consequence, computing the radius of optimal Bayesian credible ellipsoids for QSE is computationally intractable, whereas the unconstrained problem can be solved efficiently.
For Frequentist confidence regions, this result implies that computing any property of truncated confidence ellipsoids that is sensitive to truncation is hard as well.
In conclusion, although there are settings where taking into account the physical constraints of QSE drastically improves the power of the error region, doing so in an optimal way is computationally intractable.

Note that this work does not preclude the existence of algorithms for uncertainty quantification in QSE that work well-enough in practice.
Our hardness results relies on strong assumptions, some of which might be relaxed for practical applications.
For example, our results leave room for the existence of efficient approximate solutions.
Rather, our hardness result should be understood as an absolute upper bound on what such algorithms can achieve.\\


% furthemore, we propose a measruement ensemble for phase-retrieval tailored to applications in linerar optics
% in constrast to the Gaussian ensemble used in previous work, the RECR ensemble only requires preparation of 4 complex phases per mode and discrete magintudes -> able to calibrate experimetnal implemention better

In \cref{chap:phaselift}, we investigate characterizing linear optical circuits and the related phase retrieval problem.
To overcome to challenge of phase-insensitive measurements, we map the problem to rank-one matrix recovery.
By exploiting the exact rank-one constraint, we are able to perform reconstruction using an asymptotically optimal number of measurements.
Furthermore, our recovery protocol can be implemented efficiently using a positive-semidefinite program called \quotes{PhaseLift} and it is robust to noise as the rigorously proven recovery guarantees show.
From an experimentalist's point of view, characterization of linear-optical networks via PhaseLift is favourable because it reduces the number of different measurement configurations required.
As reconfiguring the chip for another input takes more time on the architecture used for the current experiment than the actual measuring process, the sample efficiency of our protocol reduces the total amount of time required.

We also propose a measurement ensemble for phase retrieval tailored to the application in optics.
In contrast to the Gaussian ensemble used in previous work, the RECR ensemble only necessitates the ability to prepare four complex phases per mode and discrete magnitudes.
\todo{Add phaselift conclusions}
This allows for calibrating the preparation stage more accurately, and hence, reduce the total error due to a mismatch of theoretical and implemented input vectors.\\



The problem of low-rank tensor reconstruction shows that in some cases constraints are necessary for an efficient solution.
High-order tensors are hard to deal with computationally due to the exponential scaling of the number of parameters.
This motivated the development of different tensor formats such as the MPS format considered here, which by construction reflects the correlation structure of certain tensors occurring in applications.
Therefore, they allow for an efficient representation of these relevant tensors.
Here, we answer the question whether such a tensor with efficient MPS representation can be reconstructed from few linear measurements.
In contrast to previous work, we are interested in both sample efficiency and computational complexity.
Hence, we consider product measurements, which are efficiently representable as well.

% finally demonstrate recovery of higher order tensors as well numerically
The analysis of the ALS algorithm yields a sufficient condition that guarantees successful recovery of any rank-one tensor using rank-one measurements.
As a prototypical example, we consider Gaussian product measurements, which are numerically shown to satisfy these conditions for a large variety of parameters.
Additionally, numerical reconstruction experiments show that we are able to reconstruct large tensors from serenely under sampled measurements.


The ALS algorithm using rank-one measurements combines both sample and computational efficiency for tensor reconstruction.
By exploiting the low-rank constraint we obtain an exponential improvement for the sampling rate compared to na\"ive approaches under the assumptions stated.
This reduction of the number of measurements necessary for recovery is also crucial for making the reconstruction computationally efficient.
The reduction of sample and computational complexity for low-rank tensor recovery is therefore qualitatively different from existing work on compressive sensing and low-rank matrix recovery.
For the latter, reconstruction without additional constraints is still feasible and, at least in the sense of polynomial scaling, still efficient.
Future work is necessary to prove the recovery condition for certain measurement ensembles.
The most promising -- as the numerical investigation in this chapter shows -- are Gaussian product measurements.


%%%%%%%%%%%%%%%%%%%%%%%%%%%%%%%%%%%%%%%%%%%%%%%%%%%%%%%%%%%%%%%%%%%%%%%%%%%%%%%
\appendix

\chapter{Appendix}%
 % -*- root: ../thesis.tex -*-
%%%%%%%%%%%%%%%%%%%%%%%%%%%%%%

\section{Generalized Bloch representation}
\label{sec:error.parametrisation}

Here, we provide the particular generalizations $\sigma_{i}$ of the Pauli matrices used in Sec.~\ref{sub:ortho.linear_inversion}.
These are exactly the generators of the group $\mathrm{SU}(d)$, see e.g.~\cite{Kimura_2003_Bloch,Byrd_2003_Characterization} for more details.
Denote by  ${\{\ket{i}\}}_i$ an orthonormal basis and let
\begin{align*}
  \Xi_{jk}^{(\textrm{Re})} &=  \ket{j}\bra{k} + \ket{k}\bra{j}, \\
  \Xi_{jk}^{(\textrm{Im})} &= -\ii \Big(\ket{j}\bra{k} - \ket{k}\bra{j}\Big), \\
  \Xi_{l}^{(\textrm{diag})} &= \sqrt{\frac{2}{l\left(l+1\right)}}\left(\sum_{j=1}^{l} \ket{j}\bra{j} - l \ket{l+1}\bra{l+1} \right).
\end{align*}
We now define the generalized Pauli matrices in terms of these auxiliary matrices:
\begin{align}
  \label{eq:parametrisation.x}
  \left\{ \sigma_{i}:i=1,\ldots,i_{d}\right\} &= \left\{ \Xi_{jk}^{(\textrm{Re})}:1\leq j<k\leq d\right\},  \\
  \label{eq:parametrisation.y}
  \left\{ \sigma_{i}:i=i_{d}+1,\ldots,2i_{d}\right\} &= \left\{ \Xi_{jk}^{(\textrm{Im})}:1\leq j<k\leq d\right\}, \\
  \label{eq:parametrisation.z}
  \left\{ \sigma_{i}:i=2i_{d}+1,\ldots,d^{2}-1\right\} &= \left\{ \Xi_{l}^{(\textrm{diag})}:1\leq l\leq d-1\right\},
\end{align}
where $i_{d}=d(d-1)/2$.
Note that the elements of the sets in Eq.~\eqref{eq:parametrisation.x}, \eqref{eq:parametrisation.y}, and \eqref{eq:parametrisation.z} generalize the Pauli matrices $\sigma_\mathrm{X}$, $\sigma_\mathrm{Y}$, and $\sigma_\mathrm{Z}$, respectively.
Since only this structure is crucial to our proof, the order of the elements in Eq.~\eqref{eq:parametrisation.x}--\eqref{eq:parametrisation.z} in arbitrary, and hence, the definition in terms of sets is well defined for our purposes.

 % -*- root: ../thesis.tex -*-
%%%%%%%%%%%%%%%%%%%%%%%%%%%%%%%%%%%%%%%%%%%%%%%%%%%%%%%%%%%%%%%%%%%%%%%%%%%%%%%%

\chapter{Phaselift}
\label{cha:phaselift_appendix}

\section{Experimental Details}%
\label{sec:pl.experimental_details}

\subsection{Photon source}
A pulsed, Titanium:Sapphire laser (Coherent Chameleon) is used to generate 140fs pulses at $\sim808$nm at a repetition rate of $80$MHz.
A half-wave plate and polarising beamsplitter (PBS) are used to attenuate the power.
Next, a $\beta$-barium borate (BBO) crystal is used to perform second harmonic generation.
Dichroic mirrors remove the remaining 808nm light and a 0.5mm Bismuth Triborate BiB$_3$O$_6$ (BiBO) crystal is used to perform spontaneous parametric down-conversion (SPDC) from the up-converted 404nm pulse.
Down-converted photons are emitted in a cone at opening angle $\theta = 6^{\circ}$ and pass through a 3nm interference filter at 808nm.
Prisms are aligned to couple light from opposite points on the SPDC cone into polarisation maintaining fibres (PMF).

When either pair is connected directly to the detectors, the ratio between the coincidence detection rate and the single detection rate is $\sim12\%$.
Taking into account the detector efficiency, this translates to a heralding efficiency of around 24$\%$.

\subsection{Integrated Circuit}
The silica-on-silicon integrated photonic chip was fabricated at the Nippon Telegraph and Telephone company (NTT) in Japan.
Flame hydrolysis deposition followed by photolithographic and reactive ion etching was used to fabricate germanium doped silica (SiO2-GeO2) waveguides with dimensions 3.5$\mu$m $\times$ 3.5$\mu$m with a silica cladding onto a silicon substrate.
Thin-film Tantalum Nitride (Ta$_2$N) thermo-optic heaters were then fabricated on top of the circuit with dimensions 1.5mm $\times$ 50$\mu$m.
The circuit is formed of a cascaded array of 30 directional couplers (each with a length of 500$\mu$m) and 30 phase shifters designed to perform a universally reconfigurable transfer matrix on six waveguide modes.

The coupling losses have been estimated as $\sim 9\%$ per facet and the directional couplers at $<2.3\%$.
The average loss fibre-to-fibre was measured to be $\sim 42\%$.
The device is actively cooled via a Peltier cooling unit.

Thermo-optic modulators are driven by electronic heater driver boards designed in-house which can deliver up to 20V with 4.9mV resolution and current up to 100mA.
These are then interfaced with a computer to set all the heaters to implement a given transfer matrix.

\subsection{Photon Detectors}
The detection system uses 6 SPADs (Perkin Elmer SPCM-AQRH-14), each with efficiencies $50-60\%$, a dark count rate of $\sim 100$Hz, timing jitter of $\sim 350$ps and a dead time of 32ns.
A coincidence counting card time-tagging all simultaneous channels in a time window usually set to be around 2ns is used to register detection events.
For each channel it is possible to set a specific time delay that is used by the counting card to compensate for the discrepancy in the signals arrival time introduced in the experiment by optical fibres, detectors, electronics and coaxial cables.

The detector efficiencies were estimated as follows.
Light was injected into the top mode of the circuit and counts were collected for 100 Haar-random unitary configurations of the circuit.
The set of relative efficiencies that minimised the sum of the total variation distances of the measured distributions to their targets was then used.
Experimental counts are adjusted by these estimated relative efficiencies.

\subsection{Reference Reconstructions}

Since our goal is to benchmark the PhaseLift characterisation technique, and not the performance of the optical chip, we compare the experimental reconstructions to reconstructions obtained with a different technique.
These reference reconstructions are obtain in two steps.
First, we estimate the absolute value of each component from single photon data:
From \cref{eq:intensities}, we see that by inserting single photons into the $k$-th input of the device -- that is choosing the standard basis vectors as inputs $\alpha =  e_k$ as inputs --  we can estimate $\abs{M_{i,k}}$.
For each input port, the counts of each detector are normalised to take into account the detector efficiencies and then divided by the total of the counts in all detectors.
The square roots of these numbers are used as the estimated amplitudes of the matrix elements.
Second, we estimate the phase of each component using HOM-dips~\cite{Hong_1987_Measurement}, following a similar approach to~\cite{laing_super-stable_2012, Dhand2016}.
However, this second step is time-consuming and only reliable for small devices.
Therefore, for the larger devices, we only perform the first step and compare only the magnitudes of the matrix elements of the reconstructions.

To perform the HOM experiments, the preparation-stage MZIs marked red in \cref{fig:experimental.schematic} were set to perform an identity transformation and both fibres carrying the photon pairs generated by the source were connected to input ports of the chip.
For each pairwise combination of the input modes of our matrices, we recorded the twofold coincidences among our detectors while changing the time delay of a photon relative to the other via a motorised translation stage.
The difference in coincidence counts as a function of delay were fit to the function:
\begin{equation}
  f(\tau)=\left(1-c_1\exp[-((\tau - c_2)/c_3)^2]\mathrm{sinc}[(\tau - c_2)/c_6]\right)(c_4\tau + c_5 + c_7 \tau^2)
\end{equation}
where $\tau$ is the time delay of the photon and $\{c_i\}$ fit parameters.
The first term approximates the temporal envelope of a Gaussian photon subject to a top-hat filter and the second term adjusts for the decoupling resulting from the movement of the translation stage.
From the coefficient $c_1$ of this fit we retrieve a bare visibility of the two photon interference.
These bare visibilities are then rescaled by a reference visibility that accounts for the partial-distinguishability of the photons in their remaining degrees of freedom.
This was periodically assessed by implementing an effective 50:50 beam splitter on the chip and observing dip visibilities ranging from 0.965 and 0.985.

In order to retrieve the phase information of the matrix elements, the two photon interference fringes with the best estimates of the visibility were selected for each matrix to create a sequence that suffices to determine the phases of all the elements of the matrix.
Assuming that all the elements of the first row and of the first column are real and positive, the complex argument of the first phase factor retrieved associated to the element $M_{ij}$ is given by the following equation.
\begin{equation}
\phi_{ij}=\pm  \arccos\left( - \frac{|M_{1,1}M_{i,j}|^2+|M_{1,j}M_{i,1}|^2}{2 |M_{1,1}M_{i,j}M_{1,j}M_{i,1}|} V_{1,j,1,i} \right)
\end{equation}
where $V_{1,j,1,i}$ is the visibility observed injecting photons into the ports 1 and j and detecting photons at the output ports 1 and i.
The sign $\pm$ in the equation is chosen to maximise the similarity to the target matrix.
Then, if we want to use a visibility $V_{\beta,j,\alpha,i}$ to deduce the phase $\phi_{ij}$ knowing $\phi_{\alpha \beta}$, $\phi_{\alpha j}$ and $\phi_{i \beta}$ we have
\begin{equation}
\phi_{ij}=\pm   \arccos\left( - \frac{|M_{\alpha \beta}M_{i,j}|^2+|M_{\alpha,j}M_{i,\beta}|^2}{2 |M_{\alpha,\beta}M_{i,j}M_{\alpha,j}M_{i,\beta}|} V_{\beta,j,\alpha,i} \right) +\phi_{\alpha j} +\phi_{i \beta} -\phi_{\alpha \beta}
\end{equation}

\subsection{Data Analysis}%
\label{sub:experimental_details.data}

\begin{table}
  \begin{tabular}{l | r r r}
    Dimension $n$ & 2 & 3 & 5 \\
    Gaussian & 20 & 30 & 40 \\
    RECR & 6 & 31 & 39 \\
  \end{tabular}
  \caption{%
    \label{tab:measurements}
    Total number of preparation vectors taken during experiment.
  }
\end{table}

As mentioned in the main text, we estimate the intensity measurements from single photon counting rates.
After correcting for detector efficiency, all counting rates are scaled by a constant such that the resulting intensities obey $\max_l \sum_j I_j(\alpha^{(l)}) = 1$.
This only amounts to scaling the transfer matrix by a constant, which does not influence the end result since we later rescale the obtained reconstruction appropriately (see \cref{eq:experimental_details.data.scaling}).
However, this simple rescaling helps with numerical stability in the SDP solver.
We provide a ready for use implementation of the PhaseLift convex program~\eqref{eq:PhaseLift} as well as related algorithms in the open source library \textsc{pypllon}~\cite{Suess_2017_Pypllon},

The post-processing of a reconstruction ${M}^\sharp$ consists of two steps:
First, we rescale the reconstruction by a constant such that
\begin{equation}
  \label{eq:experimental_details.data.scaling}
  \max_i \norm{(M^\sharp)_i}_{\ell_2} = 1,
\end{equation}
where $(M^\sharp)_i$ denotes the $i$-th row of $M^\sharp$.
In an ideal experiment, $M^\sharp$ would be unitary and, therefore, every row would have unit norm.
However, due to loss in the characterised circuit as well as detector inefficiencies, the norm of each row is smaller than one.
Since we cannot distinguish the two sources of loss in our current experimental setup, we cannot characterise the absolute photon loss in the circuit, but only the relative losses of the rows.
Estimating the dark counts in future experiments would enable characterising the absolute photon loss in the circuit as well.

The second post-processing step consists of fixing phases of the reconstructions:
Recall that we are only able to recover the transfer matrix up to its row phases since the global phases of the rows are lost in the intensity measurements.
Therefore, we fix the row phases of the PhaseLift reconstructions in \cref{fig:experimental.targetref} by minimizing the Frobenius distance to the target unitary and compute the error as
\begin{equation}
  \min_{{\mu}: \abs{\mu_i} = 1}\left\|  {M}_\mathrm{target} -  \mathrm{diag} ({\mu}) {M}^\sharp) \right\|_2
\end{equation}
However, since the HOM-dip reconstruction is insensitive to global phases of the columns as well, we have to minimize both row and column phases for the HOM-dip reconstructions in \cref{fig:experimental.targetref}.
Furthermore, since in \cref{fig:experimental.overview} the HOM-dip reconstruction is taken as reference value, we have to minimize the row and column phases for all PhaseLift reconstructions in that picture as well.
The raw data as well as the analysis scripts are available at \url{https://github.com/dseuss/phaselift-paper}.


%%%%%%%%%%%%%%%%%%%%%%%%%%%%%%%%%%%%%%%%%%%%%%%%%%%%%%%%%%%%%%%%%%%%%%%%%%%%%%%%
\section{Recovery guarantee for \cref{prot:characterization}}%
\label{sec:pl.guarantee}



%%%%%%%%%%%%%%%%%%%%%%%%%%%%%%%%%%%%%%%%%%%%%%%%%%%%%%%%%%%%%%%%%%%%%%%%%%%%%%%%
%%%%%%%%%%%%%%%%%%%%%%%%%%%%%%%%%%%%%%%%%%%%%%%%%%%%%%%%%%%%%%%%%%%%%%%%%%%%%%%%

\section{Proof of \cref{thm:phaselift_noisy}}
\label{sec:pl.main_proof}

Our analysis is inspired by Ref.~\cite{dirksen_gap_2015} (who derived strong results for sparse vector recovery using similar assumptions) and Ref.~\cite{kabanava_stable_2016} in the non-commutative setting. Moreover, Krahmer and Liu considered a real-valued version of the problem addressed here, see Ref.~\cite{krahmer_phase_2017}.




\subsection{Mathematical preliminaries}

Our analysis is based on two strong results about random matrix theory. First, the assumption of subgaussian tails \eqref{eq:subexponential} implies strong bounds on the operator norm of matrices of the form $\sum_{k=1}^m \ket{{\alpha_k}}\bra{\alpha_k}$:

\begin{theorem}[Variant of Theorem 5.35 in \cite{Vershynin_2010_Introduction}] \label{thm:bernstein}
Suppose that $\alpha_1,\ldots,\alpha_m$ are independent instances of a subgaussian random vector obeying \eqref{eq:subexponential} with constant $C_{SG}$.
Set
\begin{equation}
  \tilde{ H} = \frac{1}{m} \sum_{k=1}^m \left( a_k |\alpha_k \rangle \! \langle \alpha_k| - \mathbb{E} \left[ a_k |\alpha_k \rangle \! \langle \alpha_k| \right] \right),
  \label{eq:Htilde}
\end{equation}
where $a_k \in \mathbb{C}$ and $\abs{a_k} \leq 1$.
Then,
\begin{align*}
\mathrm{Pr} \left[ \| \tilde{ H} \|_\infty \geq t \right]
\leq
\begin{cases}
2 \exp \left( 2 \ln (3) n  - \frac{mt^2}{8 C_{SG}} \right) & 0 \leq t \leq 2C_{SG}, \\
2 \exp \left( 2 \ln (3) n - \frac{m}{2} (t- C_{SG} )  \right) & t \geq 2 C_{SG}.
\end{cases}
\end{align*}
\end{theorem}


The second result is a generalization of ``Gordon's escape through a mesh''-Theorem \cite{gordon_milman_1988} (a random subspace avoids a subset provided the subset is small in some sense) that is due to Mendelson \cite{mendelson_learning_2015,koltchinskii_bounding_2015}, see also see also \cite{tropp_convex_2015}.

\begin{theorem}[Mendelson's small ball method] \label{thm:mendelson}
  Suppose that the measurement operator $\mathcal{A}:\Hermitian_n \to \mathbb{R}^m$ contains $m$ independent copies $ A_k$ of a random matrix $ A \in \Hermitian_n$, that is
  \begin{equation}
    \label{eq:measurement_operator_definition}
    \mathcal{A}( Z) = \sum_{k=1}^m \tr ( A_k  Z) \,  e_k,
  \end{equation}
  and let $D \subset \Hermitian_n$.
  For $\xi >0$ define
  \begin{align}
    Q_\xi (D,  A) =& \inf_{ Z \in D}\mathrm{Pr} \left[ | \tr ( A_k  Z) | \geq \xi \right] \quad &\textrm{(marginal tail funtion)}, \label{eq:marginal_tail_function}\\
    W_m (D,  A) =& 2 \mathbb{E} \left[ \sup_{ Z \in D} \tr \left(  Z  H \right) \right] \quad &\textrm{(mean empirical width)},
  \end{align}
  where
  \begin{equation}
     H= \frac{1}{\sqrt{m}} \sum_{k=1}^m \eta_k  A_k
  \end{equation}
  and the $\eta_1,\ldots,\eta_m$ are independent Rademacher random variables.
  Then for any $\xi >0$ and $t >0$
  \begin{equation}
    \frac{1}{\sqrt{m}}\inf_{ Z \in D} \| \mathcal{A}( Z) \|_{\ell_1} \geq \xi \sqrt{m} Q_{2\xi}(D,  A) -  W_m (D,  A)-\xi t \label{eq:mendelson}
  \end{equation}
  with probability at least $1-\mathrm{e}^{-2t^2}$.
\end{theorem}

Note that the measurement operator introduced in \cref{eq:measurement_operator_definition} is a shorthand notation for the linear measurements $y_k = \tr  A_k  Z$ with $k=1,\ldots,m$.
It maps the signal matrix $ Z$ to the vector of (noiseless) measurement outcomes $\sum_k y_k  e_k$.


\subsection{Convex geometry}

This section summarizes several results presented in Ref.~\cite{kabanava_stable_2016} and adapts them to the task at hand: phase retrieval.
Compared to~\cite{kabanava_stable_2016} the analysis presented here is somewhat more direct and exploits the positive semidefinite constraint in a different way.

\begin{proposition} \label{prop:nsp_implication}
  Let $\Sphere^{n^2-1}=\left\{  Z \in \Hermitian_n: \|  Z \|_2=1 \right\}$ be the (Frobenius norm) unit sphere in $\Hermitian_n$ and $\mathcal{B}_1 = \mathrm{conv} \left\{ \pm | x \rangle \! \langle  x| \colon  x \in \Sphere^{n-1} \right\}$ denote the trace-norm ball.
  Define
  \begin{equation}
    D := \Sphere^{d^2-1} \cap 3 \mathcal{B}_1, \label{eq:D}
  \end{equation}
  and let $\mathcal{A}( Z) = \sum_{k=1}^m \tr ( A_k  Z ) \,  e_k$ be a measurement operator that obeys
  \begin{align}
      \frac{ \tau}{m} \| \mathcal{A}( Z) \|_{\ell_1} \geq& \norm{ Z}_2 \quad \forall  Z \in D \label{eq:nsp}\\
      \| \frac{1}{\nu m}\sum_{k=1}^m  A_k -  \mathbb{I} \|_\infty \leq& \frac{1}{6}\label{eq:approx_povm}
    %
  \end{align}
  for some $\tau,\nu >0$.
  Then, the following relation holds for any $ Z \geq 0$ and any $|{x} \rangle \! \langle {x}|$:
  \begin{equation}
    \|  Z - |{x} \rangle \! \langle {x}| \|_2 \leq \frac{1}{m} \max \left\{ \tau, \frac{6}{\nu} \right\}  \| \mathcal{A}( Z-|{x} \rangle \! \langle {x}|) \|_{\ell_1}. \label{eq:rec_guarantee}
  \end{equation}
\end{proposition}



\begin{proof}
In the proof we will frequently use the decomposition $ Z =  Z_1+ Z_c$ for $ Z$ with eigenvalue decomposition $ Z = \sum_{k=1}^n \lambda_k | z^{(k)} \rangle \! \langle  z^{(k)}|$.
Then, $ Z_1 = \lambda_1 | z^{(1)} \rangle \! \langle  z^{(1)}|$ is the leading rank-one component and $ Z_c =  Z- Z_1$ is the ``tail''.
Note that, in particular, $ Z =  Z_1$ if and only if $ Z$ has unit rank.
Fix $ Z \geq 0$ and $|{x} \rangle \! \langle {x}|$.
\Cref{eq:rec_guarantee} is invariant under re-scaling, so we may w.l.o.g.\ assume $\|  Z-|{x} \rangle \! \langle {x}|\|_2=1$.
We treat the following two cases separately:
\begin{align}
I.) \quad& \| ( Z-|{x} \rangle \! \langle {x}|)_1 \|_1 \geq \frac{1}{2} \| ( Z-|{x} \rangle \! \langle {x}|)_c \|_1, \label{eq:nsp_case1} \\
II.) \quad & \| ( Z-|{x} \rangle \! \langle {x}|)_1 \|_1 < \frac{1}{2} \| ( Z-|{x} \rangle \! \langle {x}|)_c \|_1. \label{eq:nsp_case2}
\end{align}
Note that I.) implies
\begin{align*}
\|  Z-|{x} \rangle \! \langle {x}| \|_1 \leq &\| ( Z-|{x} \rangle \! \langle {x}|)_1 \|_1 + \| ( Z-|{x} \rangle \! \langle {x}|)_c \|_1 \leq 3 \| ( Z-|{x} \rangle \! \langle {x}|)_1 \|_1 \\
 = & 3 \| ( Z-|{x} \rangle \! \langle {x}|)_1 \|_2 \leq 3 \|  Z- |{x} \rangle \! \langle {x}| \|_2 = 3
\end{align*}
which in turn implies that $ Z-| {x} \rangle \! \langle {x}|$ is contained in $3 \mathcal{B}_1$.
Thus, \eqref{eq:nsp} is applicable and yields
\begin{equation*}
\|  Z - |{x} \rangle \! \langle {x}| \|_2 \leq  \frac{\tau}{m} \| \mathcal{A}( Z-|{x} \rangle \! \langle {x}|) \|_{\ell_1}
\end{equation*}
which establishes \cref{eq:rec_guarantee} for case I in \eqref{eq:nsp_case1}.


For the second case, we use a consequence of von Neumann's trace inequality, see e.g. \cite[Theorem~7.4.9.1]{horn_topics_1991}: Let $ A,  B$ be matrices with singular values $\sigma_k ( A),\sigma_k ( B)$ arranged in non-increasing order.
Then
\begin{equation*}
  \|  A -  B \|_1 \geq \sum_{k=1}^d | \sigma_k ( A) - \sigma_k ( B)|
\end{equation*}
This relation implies
\begin{align*}
  \|  Z \|_1 =& \| |{x} \rangle \! \langle {x}| - (|{x} \rangle \! \langle {x}|- Z) \|_1
  \geq \sum_{k=1}^d \left| \sigma_k (| x \rangle \! \langle  x|) - \sigma_k (| x \rangle \! |\langle  x|-  Z ) \right| \\
  \geq & \sigma_1 (| x \rangle \langle  x|) - \sigma_1 \left( | x \rangle \! \langle  x| -  Z \right)+ \sum_{k=2}^d \sigma_k \left( | x \rangle \! \langle  x| -  Z\right) \\
  =&  \| | x \rangle \! \langle  x| \|_1  - \| (| x \rangle \! \langle  x| -  Z)_1 \|_1 + \|(| x \rangle \! \langle  x| - Z)_c \|_1 \\
  >& \| | x \rangle \! \langle  x| \|_1 + \frac{1}{2} \| (| x \rangle \! \langle  x|- Z)_c \|_1,
\end{align*}
where the last inequality follows from \eqref{eq:nsp_case2}. Consequently,
\begin{align}
  \| | x \rangle \! \langle  x| -  Z \|_1
  =& \| (| x \rangle \! \langle  x| -  Z)_1 \|_1 + \| (| x \rangle \! \langle  x|- Z)_c \|_1
  \leq \frac{3}{2} \| (| x \rangle \! \langle  x|-  Z )_c \|_1 \nonumber \\
  < & 3 \left( \|  Z \|_1 - \| | x \rangle \! \langle  x| \|_1 \right). \label{eq:nsp_aux2}
\end{align}
Now, positive semidefiniteness of both $ Z$ and $\ket{ x}\bra{ x}$ together with assumption~\eqref{eq:approx_povm} implies
\begin{align*}
  \|  Z \|_1 - \| |{x} \rangle \! \langle {x}| \|_1
  =& \tr ( Z-|{x} \rangle \! \langle {x}|) =  \tr \left( \mathbb{I} \left(  Z-| {x} \rangle \! \langle x|\right) \right) \\
  =&  \tr \left( \left( \mathbb{I} - \frac{1}{\nu m} \sum_{k=1}^m A_k \right)  Z-|{x} \rangle \! \langle {x}| \right) + \frac{1}{\nu m} \sum_{k=1}^m \tr \left( A_k ( Z-|{x} \rangle \! \langle {x}|) \right) \\
  \leq &  \left\|\mathbb{I}- \frac{1}{ \nu m} \sum_{k=1}^m A_k \right\|_\infty \|  Z-|{x} \rangle \! \langle {x}| \|_1 + \frac{1}{\nu m} \| \mathcal{A}(|{x} \rangle \! \langle {x}|- Z) \|_{\ell_1} \\
  \leq &  \frac{1}{6 } \|  Z-|{x} \rangle \! \langle {x}| \|_1 + \frac{1}{\nu m} \| \mathcal{A}(|{x} \rangle \! \langle {x}|- Z) \|_{\ell_1}.
\end{align*}
Inserting this into \eqref{eq:nsp_aux2} yields
\begin{align*}
\| | x \rangle \! \langle  x| -  Z \|_1 < \frac{1}{2} \| |{x} \rangle \! \langle {x}|- Z \|_1 +  \frac{3}{\nu m} \| \mathcal{A}(|{x} \rangle \! \langle {x}|- Z) \|_{\ell_1}
\end{align*}
which implies the claim for case II in \eqref{eq:nsp_case2}.
\end{proof}


\begin{proposition} \label{prop:RECR_nsp}
  Under the assumptions of \cref{thm:phaselift_noisy}, the measurement operator
  \begin{equation}
    \label{eq:measurement_operator_rank1}
    \mathcal{A}( Z) = \sum_k \tr \left(\ket{ a_k}\bra{ a_k}  Z\right)  e_k
  \end{equation}
  obeys both condition~\eqref{eq:nsp} and \eqref{eq:approx_povm} with probability at least $1- 3\mathrm{e}^{-\gamma m}$, provided that $C >1$ is sufficiently large.
\end{proposition}

We postpone the proof of this statement to \cref{sec:pl.proof_measurement_operator_is_good} and directly derive \cref{thm:phaselift_noisy} -- which constitutes the main theoretical achievement of this work -- from this statement.

\begin{proof}[Proof of \cref{thm:phaselift_noisy}]
\Cref{prop:RECR_nsp} implies that a measurement operator~\eqref{eq:measurement_operator_rank1} containing $m \geq C n$ measurements sampled from a distribution satisfying \eqref{eq:tight_frame}, \eqref{eq:sub_isotropy} and \eqref{eq:subexponential} meets the requirements of \cref{prop:nsp_implication} with probability at least $1-3 \mathrm{e}^{-\gamma m}$.
Conditioned on this event, we have
\begin{equation}
\|  Z - |{x} \rangle \! \langle {x}| \|_2 \leq \frac{C'}{2m}  \| \mathcal{A}( Z - |{x} \rangle \! \langle {x}|) \|_{\ell_1} \quad \forall  Z \geq 0,\; \forall {x} \in \mathbb{C}^n,
\label{eq:nsp_implication2}
\end{equation}
where $C' = 2 \max \left\{\tau, 6/\nu \right\}$.
Now, suppose that we want to reconstruct a particular ${x}$ from noisy measurements of the form ${y} = \mathcal{A} (|{x} \rangle \! \langle {x}|) + {\epsilon}$. Then Eq.~\eqref{eq:nsp_implication2} implies
\begin{align*}
\|  Z - |{x} \rangle \! \langle {x}| \|_2 \leq \frac{C'}{2m} \| \mathcal{A}( Z) - {y} + {\epsilon} \|_{\ell_1}
\leq \frac{C'}{2m} \left( \| {\epsilon} \|_{\ell_1} + \| \mathcal{A}( Z) - {y} \|_{\ell_1} \right)\quad \forall  Z \geq 0.
\end{align*}
PhaseLift -- the convex optimization problem \eqref{eq:PhaseLift} -- minimizes the right hand side of this bound over all $ Z \geq 0$. Since $ Z = |{x} \rangle \! \langle {x}|$ is a feasible point of this optimization, we can conclude that the minimizer $ Z^\sharp$ obeys
\begin{equation*}
\| \mathcal{A}( Z^\sharp) - {y} \|_{\ell_1} \leq \| \mathcal{A}(|{x} \rangle \! \langle {x}|)-{y} \|_{\ell_1} = \| {\epsilon} \|_{\ell_1}
\end{equation*}
which yields the bound presented in \eqref{eq:noisy_recovery_bound}.
\end{proof}


\section{Proof of \cref{prop:RECR_nsp}}
\label{sec:pl.proof_measurement_operator_is_good}




\begin{lemma}[Bound on the marginal tail function]
  Let $D$ be the set introduced in \eqref{eq:D} and let $ A =| a \rangle \! \langle  a| $, where $ a$ satisfies \eqref{eq:sub_isotropy} and \eqref{eq:subexponential}.
  Then, the marginal tail function \eqref{eq:marginal_tail_function} obeys
  \begin{equation*}
    Q_\xi (D,  A) \geq  C_Q \left( 1-  \frac{\xi^2}{C_{SI}}\right)^2  \quad \forall 0 \leq \xi \leq \sqrt{C_{SI}},
  \end{equation*}
  where $C_Q>0$ is a sufficiently small constant.
\end{lemma}

\begin{proof}
Fix $ Z \in D$, then $\|  Z \|_2 =1$ by definition of $D$.
Note that sub-isotropy \eqref{eq:sub_isotropy} and the Paley-Zygmund inequality imply for any $\xi \in [0,1]$
\begin{align*}
  \mathrm{Pr} \left[ | \langle  a|  Z | a \rangle| \geq \xi \right]
  \geq & \mathrm{Pr} \left[ \langle  a|  Z | a \rangle^2 \geq \frac{\xi^2}{C_{SI}} \mathbb{E} \left[ \langle  a| Z| a \rangle^2 \right] \right]
  \geq \left(1-\frac{\xi^2}{C_{SI}}\right)^2 \frac{\mathbb{E} \left[ \langle  a | Z | a \rangle^2 \right]^2}{\mathbb{E} \left[ \langle  a|  Z | a \rangle^4 \right]}.
\end{align*}
Sub-isotropy ensures that the numerator is lower bounded by $C_{SI}^2 \|  Z \|_2^4 = C_{SI}^2$.
In order to derive an upper bound on the denominator, we use the constraint $\|  Z \|_1 \leq 3$ for any $ Z \in D$ together with the subgaussian tail behavior \eqref{eq:subexponential} of $ a$.
Insert an eigenvalue decomposition $ Z = \sum_{i=1}^n \lambda_i | z^{(i)} \rangle \! \langle  z^{(i)}|$ (with $\lambda_i \in \mathbb{R}$ and $ z^{(i)} \in \Sphere^{n-1}$) and note
\begin{align}
  \mathbb{E} \left[ \langle  a|  Z | a \rangle^4 \right]
  \leq & \sum_{i_1,i_2,i_3,i_4=1}^n | \lambda_{i_1} \lambda_{i_2} \lambda_{i_3} \lambda_{i_4} | \mathbb{E} \left[ \prod_{k=1}^4 | \langle  a,  z^{(i_k)} \rangle|^2 \right]. \label{eq:Q_aux1}
\end{align}
Now fix $ z^{(i_1)},\ldots, z^{(i_4)}$ and use a combination of the AM-GM inequality and the fundamental relation between $\ell_p$-norms ($\|  v \|_{\ell_1} \leq k^{1-\frac{1}{k}} \|  v \|_{\ell_k}$ for $v \in \mathbb{R}^k$) to conclude
\begin{align*}
  \mathbb{E} \left[ \prod_{k=1}^4 | \langle  a, z^{(i_k)}\rangle |^2 \right]
  \leq \frac{1}{4} \sum_{k=1}^4 \mathbb{E} \left[ | \langle  a,  z^{(i_k)} \rangle|^8 \right]
  \leq C_{SG} 4!,
\end{align*}
where the last inequality follows from condition \eqref{eq:subexponential}.
Consequently,
\begin{align*}
  \mathbb{E} \left[ \langle  a|  Z |  a\rangle^4 \right]
  \leq C_{SG} 4! \sum_{i_1,i_2,i_3,i_4} | \lambda_{i_1} \lambda_{i_2} \lambda_{i_3} \lambda_{i_4} |
  = 24 C_{SG} \|  Z \|_1^4 \leq 24*3^4 C_{SG},
\end{align*}
because $ Z \in D$ implies $\|  Z \|_1 \leq 3$.
In summary,
\begin{align*}
  \mathrm{Pr} \left[ | \langle  a|  Z | a \rangle| \geq \xi \right]
  \geq \left(1-\frac{\xi^2}{C_{SI}}\right)^2 \frac{\mathbb{E} \left[ \langle  a|  Z | a \rangle^2 \right]^2}{\mathbb{E} \left[ \langle  a|  Z | a \rangle^4 \right]}
  \geq \left(1-\frac{\xi^2}{C_{SI}}\right)^2 \frac{C_{SI}^2}{1944C_{SG}}
\end{align*}
and the bound on $Q_\xi (D, A)$ with $C_Q = \frac{C_{SI}^2}{1944 C_{SG}}$ follows from the fact that this lower bound holds for any $ Z \in D$.
\end{proof}



\begin{lemma}[Bound on the mean empirical width]
Let $D$ be the set introduced in \eqref{eq:D} and let $ H = \frac{1}{\sqrt{m}} \sum_{k=1}^m \eta_k | \alpha_k \rangle \! \langle \alpha_k|$, where each $\alpha_k$ is subexponential in the sense of \eqref{eq:subexponential} and $m \geq \frac{2 \ln (3)}{C_{SG}} n$.
Then there exists a constant $C_W >0$ such that
\begin{equation*}
W_m (D, A) \leq C_W \sqrt{n},
\end{equation*}
\end{lemma}



\begin{proof}
%\textcolor{myblue}{Proof idea: the Bernstein typ inequality \cref{thm:bernstein} applied to $H$ and gives strong tail bounds. We use this result to bound the tail in $\mathbb{E} \left[ \| H \|_\infty \right] = \int_0^\infty \mathrm{Pr} \left[ \| H \|_\infty \geq t \right] \mathrm{d}t$
%}

Note that by construction $D \subset 3 \mathcal{B}_1$ and consequently
\begin{align}
  W_m (D,  A) = 2 \mathbb{E} \left[ \sup_{ Z \in D} \tr ( Z  H) \right] \leq 6 \mathbb{E} \left[ \sup_{ Z \in \mathcal{B}_1} \tr ( Z  H) \right] = 6 \mathbb{E} \left[ \|  H \|_\infty  \right], \label{eq:Wm_hoelder}
\end{align}
where the last equality follows from the duality of trace and operator norm. Now note that $\tilde{ H} = \sqrt{m}  H$ is of the form \eqref{eq:Htilde}, where each $a_k$ is an independent Rademacher random variable.
\cref{thm:bernstein} thus implies
\begin{align}
  \mathrm{Pr} \left[\|  H \|_\infty \geq t \right]
  %= \mathrm{Pr} \left[ \| \tilde{H} \|_\infty \geq \frac{t}{\sqrt{m}} \right]
  \leq
  \begin{cases}
   2 \times 9^n \exp \left( - \frac{t^2}{8 C_{SG}} \right) & t \leq 2C_{SG} \sqrt{m}, \\
  2 \times 9^n \exp \left( - \frac{\sqrt{m}}{2} \left( t - C_{SG} \sqrt{m} \right) \right) & t \geq 2 C_{SG} \sqrt{m}
  \end{cases}
  \label{eq:Wm_tails}
\end{align}
and we can bound $\mathbb{E} \left[ \|  H \|_\infty \right]$ by using the absolute moment formula,
%$
%\mathbb{E} \left[ \| H \|_\infty \right] = \int_0^\infty \mathrm{Pr} \left[ \| H \|_\infty \geq t \right] \mathrm{d}t,
%$
see e.g.\ \cite[Propostion~7.1]{Foucart_2013_Mathematical}, and bounding the effect of the tails via \eqref{eq:Wm_tails}.
To this end, we split the real line into three intervals $[0, c \sqrt{n}], [c\sqrt{n}, 2 C_{SG} \sqrt{m}], [2 C_{SG} \sqrt{m},\infty[$, where $c$ is a constant that we fix later:
\begin{align*}
  \mathbb{E} \left[ \| H\|_\infty \right] =& \int_0^\infty \mathrm{Pr} \left[ \| H\|_\infty \geq t \right] \mathrm{d}t \\
  \leq & \int_0^{c \sqrt{n}} 1 \mathrm{d}t + 2 \times 9^n \left( \int_{c \sqrt{n}}^{2 C_{SG} \sqrt{m}} 2 \exp \left( - \frac{t^2}{8 C_{SG}} \right) \mathrm{d}t
  +  \mathrm{e}^{\frac{m C_{SG}}{2}} \int_{2 C_{SG} \sqrt{m}}^\infty \exp\left( - \frac{\sqrt{m}t}{2}  \right) \mathrm{d} t \right)\\
  \leq & c \sqrt{n} + 2 \times 9^n \left( \int_{c \sqrt{n}}^{2 C_{SG} \sqrt{m}}  \exp \left( - \frac{t^2}{8 C_{SG}} \right) \mathrm{d}t
  + \frac{2}{\sqrt{m}} \mathrm{e}^{-\frac{C_{SG} m}{2}}\right).
\end{align*}
For the remaining Gauss integral, we use $\frac{t}{c \sqrt{n}} \geq 1\; \forall t \geq c\sqrt{n}$ to conclude
\begin{align*}
  \int_{c \sqrt{n}}^{2 C_{SG} \sqrt{m}}  \exp \left( - \frac{t^2}{8 C_{SG}} \right) \mathrm{d}t
  %\leq & \int_{C \sqrt{n}}^{4 \mathrm{e}^2 \sqrt{m}} \frac{t}{C \sqrt{n}}  \exp \left( - \frac{t^2}{32 \mathrm{e}^2} \right) \mathrm{d} t
  \leq  \int_{c \sqrt{n}}^\infty \frac{t}{c \sqrt{n}}  \exp \left( - \frac{t^2}{8 C_{SG}} \right) \mathrm{d} t
  %=& - \frac{32 \mathrm{e}^2}{C\sqrt{n}} \exp \left( - \frac{t^2}{32 \mathrm{e}^2}\right) |_{t=C \sqrt{n}}^{t=\infty}
  = \frac{8 C_{SG}}{c \sqrt{n}} \exp \left( - \frac{c^2 n}{8 C_{SG}} \right).
\end{align*}
Now, fixing $c = 4 \sqrt{\ln (3)C_{SG}}$ assures $\exp \left( -\frac{c^2 n}{8 C_{SG}}\right) = 9^{-n}$ and consequently
\begin{align*}
  \mathbb{E} \left[ \|  H \|_\infty \right]
  \leq & 4  \sqrt{ \ln (3) C_{SG} n} + \frac{4 \sqrt{C_{SG}}}{\sqrt{ \ln (3) n}} + \frac{4}{\sqrt{m}} \mathrm{e}^{2 \ln (3) n - C_{SG} m} \\
  %\leq &8 \mathrm{e} \sqrt{ \ln (3) n} + \frac{8 \mathrm{e}}{\sqrt{ \ln (3) n}} + \frac{8 \mathrm{e}}{\sqrt{4 \ln (3) n}} \\
  \leq & 4\sqrt{C_{SG}} \left( \sqrt{ \ln (3) n} + \frac{2}{\sqrt{ \ln (3) n}} \right) \leq 12 \sqrt{ \ln (3) C_{SG} n}.
\end{align*}
where the second inequality follows from $m \geq \frac{2 \ln (3)}{C_{SG}} n$. Inserting this bound into \eqref{eq:Wm_hoelder} yields the claim with $C_W = 72 \sqrt{ \ln (3) C_{SG}}$.
\end{proof}

Now we are ready to apply Mendelson's small ball method \eqref{eq:mendelson}.
For $D$ defined in \eqref{eq:D} and measurements $ A_k = |\alpha_k \rangle \! \langle \alpha_k|$ with $\alpha_k$ obeying \eqref{eq:sub_isotropy} and \eqref{eq:subexponential} the bounds from the previous Lemmas imply
\begin{align*}
  \frac{1}{\sqrt{m}}\inf_{ Z \in D} \|\mathcal{A}( Z) \|_{\ell_1} \geq \xi \sqrt{m} C_Q \left( 1- \frac{4 \xi^2}{C_{SI}} \right)^2 - 2 C_W \sqrt{n} - \xi t \quad \forall \xi \in (0, 1/\sqrt{C_{SI}}), \forall t \geq 0
\end{align*}
with probability at least $1- \mathrm{e}^{-2t^2}$. We choose $\xi = \sqrt{C_{SI}}/4$ and $t = \gamma_1 \sqrt{m}$, where $\gamma_1 = \frac{9 C_Q}{32}$ and obtain with probability at least $1-\exp \left( -2 \gamma_1 m \right)$:
\begin{align*}
  \frac{1}{\sqrt{m}}\inf_{ Z \in D} \|\mathcal{A}( Z) \|_{\ell_1} \geq & \frac{9 C_Q\sqrt{C_{SI}}}{64} \sqrt{m} -  C_W\sqrt{n} - \frac{\sqrt{C_{SI}}}{4} \frac{9 C_Q}{32} \sqrt{m} \\
  = & C_W \left( \frac{9 C_Q \sqrt{C_{SI}}}{128 C_W} \sqrt{m} - \sqrt{n} \right).
\end{align*}
Setting $m = C n$ with $C = \left( \frac{256 C_W}{9 C_Q \sqrt{C_l}} \right)^2$ implies
\begin{equation*}
  \frac{1}{\sqrt{m}} \inf_{ Z \in D} \| \mathcal{A}( Z) \|_{\ell_1} \geq 2 C_W \sqrt{n} = \frac{2 C_W}{\sqrt{C}} \sqrt{m}
\end{equation*}
with probability at least $1- \mathrm{e}^{-2 \gamma_1 m}$.
For $\tau = \frac{ 2 C_W}{\sqrt{C}}$, the first claim in \cref{prop:RECR_nsp} follows from rearranging this expression and using $\|  Z \|_2=1$ for all $ Z \in D$.

Let us now move on to establishing the second statement \eqref{eq:approx_povm}:
Isotropy \eqref{eq:tight_frame} implies
\begin{align*}
  \frac{1}{ C_I m} \sum_{k=1}^m |\alpha_k \rangle \! \langle \alpha_k | - \mathbb{I}
  = \frac{1}{C_{SG} m} \sum_{k=1}^m \left( |\alpha_k \rangle \! \langle \alpha_k| - \mathbb{E} \left[ |\alpha_k \rangle \! \langle \alpha_k| \right] \right)
\end{align*}
and each $\alpha_k$ has subgaussian tails by assumption \eqref{eq:subexponential}.
Thus, \cref{thm:bernstein} is applicable and setting $t= \min \left\{\frac{1}{6},2 C_{SG} \right\}$ yields
\begin{align*}
  \mathrm{Pr} \left[ \left\| \frac{1}{C_I m} \sum_{k=1}^m |\alpha_k \rangle \! \langle \alpha_k| -  \mathbb{I} \right\|_\infty \geq \frac{1}{6} \right]
  \leq 2 \exp \left( 2 \ln (3) n - \frac{C_I m \min\left\{ 1/6, 2 C_{SG} \right\}}{8 C_{SG}} \right) \leq 2 \exp \left( - \gamma_2 m \right),
\end{align*}
where the second inequality follows from $m \geq C n$, provided that $C$ is sufficiently large. Finally, we use the union bound  for the overall probability of failure and set $\gamma := \min \left\{ 2 \gamma_1,\gamma_2 \right\}$.



\section{Proof of \cref{prop:gauss+recr_requirements}}
\label{sec:pl.gauss+recr_requirements}

We now proof the crucial properties \eqref{eq:tight_frame}--\eqref{eq:subexponential} for the different measurement ensembles from \cref{prop:gauss+recr_requirements}.

\subsection{The Gaussian sampling scheme}


Let $\alpha \in \mathbb{C}^n$ be a standard (complex) Gaussian vector and fix any $ z \in \mathbb{C}^n$.
Then, the random variable $\langle \alpha, z \rangle$ is an instance of a standard (complex normal) random variable $a = \tfrac{\|  z \|_{\ell_2}}{\sqrt{2}} \left(a_R + i a_I\right)$ with $a_R, a_I \sim \mathcal{N}(0,1)$.
In turn, $|a|^2 = \frac{\|  z \|_{\ell_2}^2}{2} (a_R^2 + a_I^2)$ is a re-scaled version of a $\chi^2$-distributed random variable with two degrees of freeom. The moments of such a random variable are well-known and we obtain
\begin{equation}
  \mathbb{E} (| \langle \alpha, z \rangle|^{2N})= \left( \frac{ \|  z \|_{\ell_2}}{\sqrt{2}}\right)^N \times 2^N N! = \|  z \|_{\ell_2}^N N! \; .\label{eq:moments_gauss}
\end{equation}
From this, we can readily infer $C_{SG} = 1$, and the special case $N=1$  yields $C_I=1$.

For the remaining expression, use an eigenvalue decomposition $ Z = \sum_{k=1}^d \zeta_k | z^{(k)} \rangle \langle  z^{(k)}|$ (with normalized eigenvectors $ z^{(k)}\in \mathbb{C}^n$) and note that the random variables $|\langle  a, z^{(1)} \rangle|,\ldots, | \langle  a, z^{(n)} \rangle|$ are independently distributed and obey \cref{eq:moments_gauss}.
Consequently:
\begin{align*}
  \mathbb{E} \left[ \tr \left(  A  Z \right)^2 \right]
  =& \mathbb{E} \left[ \left( \sum_{k=1}^d \zeta_k | \langle \alpha, z^{(k)} \rangle|^2 \right)^2 \right] \\
  =& \sum_{k \neq l} \zeta_k \zeta_l \mathbb{E} \left[ |\langle \alpha, z^{(k)} \rangle|^2 \right] \mathbb{E} \left[ | \langle  a, z^{(l)} \rangle|^2 \right]
  + \sum_{k=1}^d \zeta_k^2 \mathbb{E} \left[ | \langle  a,  z^{(k)} \rangle|^4 \right] \\
  =& \sum_{k \neq l} \zeta_k \zeta_l \| z^{(k)} \|_{\ell_2}^2 \|  z^{(l)} \|_{\ell_2}^2 + 2 \sum_{k=1}^d \zeta_k^2 \|  z^{(k)} \|_{\ell_2}^4
  = \sum_{k,l=1}^d \zeta_k \zeta_l + 2\sum_{k=1}^d \zeta_k^2 \\
  =& \tr ( Z)^2 + \tr ( Z^2)
  \geq \|  Z \|_2^2,
\end{align*}
which implies $C_{SI} = 1$.

\subsection{The uniform sampling scheme}

Here, $\alpha$ is chosen uniformly from the complex sphere with radius $\sqrt{n}$.
This in turn implies that the distribution of $\alpha \in \mathbb{C}^n$ is invariant under arbitrary unitary transformations.
Techniques from representation theory -- more precisely: Schur's Lemma -- then imply
\begin{equation}
  \label{eq:from_schur}
  \mathbb{E} \left[ (|\alpha \rangle \! \langle \alpha| )^{\otimes N} \right] =
  %\mathbb{E} \left[ U^{\otimes N} (|a_0 \rangle \! \langle a_0|)^{\otimes N} (U^\dagger)^{\otimes N} \right] = \binom{n+N-1}{n}^{-1} \| a_0\|_{\ell_2}^N P_{\vee^N} =
  n^N \binom{n+N-1}{N}^{-1}  P_{\vee^N},
\end{equation}
see e.g.\ \cite[Lemma~1]{scott_tight_2006}.
Here, $ P_{\vee^N}$, denotes the projector onto the totally symmetric subspace $\bigvee\!^N \subset \left( \mathbb{C}^n \right)^{\otimes N}$.
Note that $\left(| z \rangle \! \langle  z| \right)^{\otimes N} \in \bigvee\!^N$ and, moreover $2 \mathrm{tr} \left(  P_{\vee^2}  Z^2 \right)= \|  Z \|_2^2 + \mathrm{tr} ( Z)^2$ for any matrix $ Z$, see e.g.\ \cite[Lemma~17]{kueng_low_2016}.
Consequently,
\begin{align*}
  \mathbb{E} \left[ | \langle\alpha, z \rangle|^2 \right]
  =& \mathrm{tr} \left( | z \rangle \! \langle  z| \, \mathbb{E} \left[ |\alpha \rangle \! \langle\alpha| \right] \right)
  = \mathrm{tr} \left( | z \rangle \! \langle  z| \mathbb{I} \right) = \|  z \|_{\ell_2}^2, \\
  \mathbb{E} \left[
  \langle\alpha| Z |\alpha \rangle^2 \right]
  =& \tr \left( \mathbb{E} \left[ (|\alpha \rangle \! \langle\alpha|)^{\otimes 2} \right]  Z^{\otimes 2} \right)
  = \frac{n}{n+1} \left( \|  Z \|_2^2 + \mathrm{tr}( Z)^2 \right) \geq \frac{n}{n+1} \|  Z \|_2^2, \\
  \mathbb{E} \left[ | \langle\alpha,  z \rangle |^{2N} \right]
  =& \mathrm{tr} \left(\mathbb{E} \left[ (|\alpha \rangle \! \langle\alpha|)^{\otimes N} \right]  (| z\rangle \! \langle  z|)^{\otimes N}  \right)
  = n^N \binom{n+N-1}{N}^{-1} \|  z \|_{\ell_2}^{2N} \\
  =& N! \frac{n^N (n-1)!}{(n+N-1)!} \leq N!,
\end{align*}
which implies $C_I=1$, $C_{SI} = \frac{n}{n+1}$ and $C_{SG}=1$.


\subsection{The RECR sampling scheme}

\begin{lemma}[The RECR ensemble is isotropic on $\mathbb{C}^n$]
Suppose that $\alpha$ is chosen from a RECR ensemble with erasure probability $1-p$. Then
\begin{align*}
  \mathbb{E} \left[ | \langle  \alpha, z \rangle|^2 \right] = p \|  z \|_{\ell_2}^2
  \quad \forall  z \in \mathbb{C}^n.
\end{align*}
\end{lemma}

\begin{proof}
Let $\alpha_k = \langle  e_k, \alpha\rangle$, where $ e_1,\ldots, e_n$ is the orthonormal basis with respect to which the RECR vector is defined.
Theses components obey $\mathbb{E}\left[ \alpha_k \right] = \mathbb{E} \left[ \cc{\alpha}_k \right] = 0$, as well as $\mathbb{E} \left[ |\alpha_k|^2 \right] = p$.
For any $ z \in \mathbb{C}^n$ we then have
\begin{align*}
  \mathbb{E} \left[| \langle  \alpha,  z\rangle |^2 \right]
  =& \sum_{i,j=1}^n \mathbb{E} \left[ \cc{\alpha}_i \alpha_j \right] \langle  e_i |  z \rangle \langle  z |  e_j \rangle = p \sum_{i=1}^n | \langle  e_i,  z \rangle|^2 = p \|  z \|_{\ell_2}^2.
\end{align*}
\end{proof}

\begin{lemma}[The RECR ensemble is sub-isotropic on $\Hermitian_n$]
  \label{lem:recr_subisotropic}
  Suppose that $\alpha$ is chosen from a RECR ensemble with erasure probability $1-p$. Then
  \begin{equation*}
  \mathbb{E} \left[ \langle  \alpha|  Z | \alpha \rangle^2 \right] \geq p \min \left\{ p, 1-p \right\} \|  Z \|_2^2 \quad \forall  Z \in \Hermitian_n
  \end{equation*}
\end{lemma}

\begin{proof}
Fix $ Z \in \Hermitian_n$ and compute
\begin{align*}
  \mathbb{E} \left[ \langle \alpha |  Z | \alpha \rangle^2 \right]
  =& \sum_{i,j,k,l} \mathbb{E} \left[ \bar{\alpha}_i \alpha_j \cc{\alpha^\prime_k} \alpha^\prime_l \right] \langle  e_i| Z|  e_j \rangle \langle  e_k | Z|  e_l \rangle \\
  =& \sum_{i} \mathbb{E} \left[ | \alpha_i |^4 \right] \langle  e_i| Z| e_i \rangle^2 + \sum_{i \neq k} \mathbb{E} \left[ | \alpha_i |^2 | \alpha_k|^2 \right] \left( \langle  e_i| Z| e_i \rangle \langle  e_k| Z| e_k \rangle + \langle  e_i| Z| e_k \rangle \langle  e_k|  Z| e_i \rangle \right) \\
  =& p \sum_{i=1}^n \langle  e_i| Z| e_i \rangle^2 + p^2 \sum_{i \neq k} \left( \langle  e_i| Z| e_i \rangle \langle  e_k| Z| e_k \rangle + \langle  e_i| Z| e_k \rangle \langle  e_k | Z|  e_i\rangle \right) \\
  =& p^2 \sum_{i,k=1}^n \left( \langle  e_i| Z| e_i \rangle \langle  e_k| Z| e_k \rangle + \langle  e_i| Z| e_k \rangle \langle  e_k | Z|  e_i\rangle \right) + p(1-2 p) \sum_{i=1}^n \langle  e_i| Z| e_i \rangle^2 \\
  =& p^2 \left( \tr ( Z)^2 + \| Z\|_2^2 \right) + p (1-2 p) \sum_{i=1}^n \langle  e_i|  Z |  e_i \rangle^2 \\
  \geq& p^2 \| Z\|_2^2 + p(1-p) \sum_{i=1}^n \langle  e_i | Z| e_i \rangle^2
\end{align*}
Finally, we make a case distinction:
\begin{itemize}
\item[$p \leq 1/2$]: This implies $p(1-2p) \geq 0$ and consequently
\begin{align*}
  \mathbb{E} \left[ \langle \alpha | Z| \alpha \rangle^2 \right] \geq p^2 \|  Z \|_2^2.
\end{align*}
\item[$p \geq 1/2$]: Use $\sum_{i=1}^n \langle i| X|i \rangle^2 \leq \| X \|_2^2$ to conclude
\begin{align*}
  \mathbb{E} \left[ \langle \alpha |  Z | \alpha \rangle^2 \right]
\geq ( p^2 - p|1-2p|) \| Z\|_2^2 = p(1-p) \|  Z \|_2^2.
\end{align*}
\end{itemize}
\end{proof}

\begin{lemma}[Subgaussian tails of the RECR distribution]
Suppose that $\alpha$ is a vector from the RECR ensemble. Then
\begin{align*}
\mathbb{E} \left[ | \langle \alpha,  z \rangle|^{2N} \right] \leq \mathrm{e}^{\frac{3}{2}} N! \quad \forall  z \in \Sphere^{n-1}.
\end{align*}
\end{lemma}

\begin{proof}
Fix $ z \in \mathbb{C}^n$ with $\|  z \|_{\ell_2}=1$ and note that $|\alpha_k| \leq 1$ together with the independence of $\alpha_k,\alpha_l$ for $k \neq l$ implies
\begin{align}
  \mathbb{E} \left[ \exp \left( | \langle \alpha,  z \rangle|^2 \right) \right]
  =& \mathbb{E} \left[ \prod_{k=1}^n \exp \left( | \alpha_k|^2 |z_k|^2 \right) \prod_{k \neq l} \exp \left( \cc{\alpha}_k \alpha_l \cc{z}_k z_l \right) \right] \nonumber \\
  \leq& \exp \left( \|  z \|_{\ell_2}^2 \right) \prod_{k \neq l} \mathbb{E} \left[  \exp \left( \cc{\alpha}_k \alpha_l \cc{z}_k z_l \right)  \right]. \label{eq:moment_aux1}
\end{align}
Now note that for $k \neq l$, $\cc{\alpha}_k \alpha_l$ is again a RECR random variable $\tilde{\alpha}_{k,l}$, but with erasure probability $1-p^2$.
Moreover, every RECR random variable $\alpha$ can be decomposed into the product of two independent random variables: $ \alpha= \eta \omega$, where $\eta$ is a Rademacher random variable and $\omega \in \left\{0, 1,i \right\}$ obeys $| \omega | \leq 1$.
Consequently
\begin{align*}
  \mathbb{E} \left[ \exp \left( \bar{\alpha}_k \alpha_l \bar{z}_k z_l \right) \right]
  =& \mathbb{E} \left[ \exp \left( \tilde{\alpha}_{k,l} \bar{z}_k z_l \right) \right]
  = \mathbb{E}_{\omega} \left[ \mathbb{E}_\eta \left[ \eta \omega \bar{z}_k  z_l \right] \right]
  = \mathbb{E}_{\omega} \left[ \cosh \left( \omega \bar{z}_k z_l \right) \right] \\
  \leq & \mathbb{E}_\omega \left[ \exp \left( |\omega \bar{z}_k z_l|^2/2 \right) \right]
  \leq  \exp \left( \frac{|z_k|^2 |z_l|^2}{2} \right),
\end{align*}
where we have used the standard estimate $\cosh (x) \leq \exp \left( |x|^2/2 \right)$ $\forall x \in \mathbb{C}$, as well as $| \omega| \leq 1$. Inserting this bound into \eqref{eq:moment_aux1} yields
\begin{align*}
  \mathbb{E} \left[ \exp \left( | \langle  \alpha,  z \rangle|^2 \right) \right]
  \leq \exp \left( \|  z \|_2^2 \right) \prod_{k \neq l} \exp \left( \frac{|z_k|^2 |z_l|^2}{2} \right)
  \leq \exp \left( \|  z \|_2^2 + \frac{1}{2}\|  z \|_{\ell_2}^4 \right) = \mathrm{e}^{\frac{3}{2}},
\end{align*}
because $\|  z \|_{\ell_2}=1$.
Markov's inequality shows that this exponential bound implies a subexponential tail bound for the random variable $| \langle  \alpha, z \rangle|^2$:
\begin{align*}
  \mathrm{Pr} \left[ | \langle \alpha, z \rangle|^2 \geq t \right]
  =& \mathrm{Pr} \left[ \exp \left( | \langle  \alpha, z \rangle|^2 \right) \geq \exp \left( t \right) \right]
  \leq \frac{ \mathbb{E} \left[ \exp \left( | \langle \alpha,  z \rangle|^2 \right) \right]}{\exp (t)} \leq \mathrm{e}^{\frac{3}{2}-t}.
\end{align*}
This in turn implies the following bound on the moments:
\begin{align*}
  \mathbb{E} \left[ | \langle  \alpha, z \rangle|^{2N} \right]
  =  N \int_0^\infty \mathrm{Pr}\left[ | \langle  \alpha, z \rangle|^2\geq t \right] t^{N-1} \mathrm{d}t \leq N \mathrm{e}^{\frac{3}{2}} \int_0^\infty \mathrm{e}^{-t} t^{N-1} \mathrm{d}t
  = \mathrm{e}^{\frac{3}{2}} N!,
\end{align*}
where we have used a well-known integration formula for moments, see e.g.\ \cite[Prop.~7.1]{Foucart_2013_Mathematical}, as well as integration by parts.
\end{proof}


\subsection{The normalised RECR scheme}
\label{sub:pl.normalized_recr}

\todo[noline]{DS/RiK: Expand on this idea -- or any other}

Let $ A_1,\ldots, A_m \in \Hermitian_n$ denote unnormalized RECR measurements.
Then, our result implies that w.h.p.\ any $ X = | x \rangle \!\langle  x|$ can be recovered from $m \geq  C n$ measurements of the form
\begin{equation*}
  y_k = \tr \left(  A_k  X \right) + \epsilon_k
\end{equation*}
via solving
\begin{equation}
  \underset{ Z\geq 0}{\textrm{minimize}} \quad \| \mathcal{A}( Z) -  y \|_{\ell_1}. \label{eq:phaselift2}
\end{equation}
The solution of this program is guaranteed to obey
\begin{align*}
\|  Z -  X \|_2 \leq \frac{C' \| \epsilon \|_{\ell_1}}{m}.
\end{align*}
Now suppose that we have $m$ normalized RECR measurements instead: $\tilde{ A}_k = \frac{n}{\|  A_k \|_{2}}  A_k$. Then the associated measurements correspond to
\begin{equation*}
  \tilde{y}_k = \tr \left( \tilde{ A}_k X \right) + \epsilon_k = \frac{ n}{\|  A_k \|_2} \tr \left(  A_k X \right) + \tilde{\epsilon}_k.
\end{equation*}
Multiplying this expression by $\frac{\|  A_k \|_2}{n}$ yields
\begin{equation*}
\underset{:= y_k}{\underbrace{ \frac{ \|  A_k \|_2}{n}\tilde{y}_k}}
= \tr \left(  A_k X \right) + \underset{:= \epsilon_k}{\underbrace{ \frac{ \|  A_k \|_2}{n}\tilde{\epsilon}_k}}
\end{equation*}
and solving \eqref{eq:phaselift2} for re-scaled measurement outcomes $y_k = \frac{ \|  A_k \|_2}{n}\tilde{y}_k$ yields an estimator of $ X$ that is guaranteed to obey
\begin{equation*}
  \|  Z -  X \|_2 \leq \frac{C' \|  \epsilon \|_{\ell_1}}{m}
  = \frac{C'}{m} \sum_{k=1}^m \frac{ \|  A_k \|_2}{n} | \tilde{\epsilon}_k |
  \leq \frac{C'}{m} \sum_{k=1}^m | \tilde{\epsilon}_k| = \frac{C' \| \tilde{\epsilon} \|_{\ell_1}}{m}.
\end{equation*}
Here, the last line is due to $\|  A_k \|_2 = \| \alpha_k \|_{\ell_2}^2 \leq n$.

In summary: normalizing the RECR measurements changes the signal-to-noise ratio $\frac{ \|  A_k \|_2}{| \epsilon_k|}$ in an advantageous way: it makes it bigger. Thus the stability guarantee of the unnormalized RECR ensemble allows us to deduce one for the normalized case as well. However, this approach requires a re-scaling of the measurements for the algorithmic reconstruction:
\begin{equation*}
  y_k = \frac{ \|  A_k \|_2}{n} \tilde{y}_k.
\end{equation*}

\end{document}

% -*- root: ../thesis.tex -*-
%%%%%%%%%%%%%%%%%%%%%%%%%%%%%%%%%%%%%%%%%%%%%%%%%%%%%%%%%%%%%%%%%%%%%%%%%%%%%%%%

\section{Tensors}%
\label{sec:tensors_appendix}


\todo{proof why RIP not working for our measurements}
\todo{Text}

Adapted from \cite[Thm. 2.1]{Taylor}
\begin{theorem}%
  \label{thm:ana.max_principle}
  Suppose $\Omega$ is an open bounded domain in $\Reals^n$.
  If $u \in C(\overline{\Omega}) \cap C^2(\Omega)$ and $\laplace u \ge 0$ on $\Omega$, then
  \[
    \sup_{x \in \Omega} u(x) = \sup_{y \in \partial\Omega} u(y).
  \]
\end{theorem}


\todo{We don't need this anymore?}
The next step is to derive a lower bound on $\kappa_1$ assuming that the overlap of the local tensors is not too small.
For this purpose, we state the following Lemma.

\begin{lemma}%
  \label{lem:ana.kappa_bound}
  Let $\alpha_i \in [0,1]$ with $i = 1, \ldots, N - 1$.
  If for $\delta \in (0, 1)$, $\prod_i \alpha_i^2 \ge 1 - \delta^2$ holds, then for any $C > 0$
  \[
    C \prod_i \alpha_i - \prod_i \sqrt{1 - \alpha_i^2} \ge C  - \delta (1 + C)
  \]
\end{lemma}
\begin{proof}
  We first show that the function
  \[
    g(\eta_1, \ldots, \eta_{N-1}) = 1 - \prod_i \eta_i - \prod_i \left( 1 - \eta_i \right)
  \]
  satisfies
  \[
    \label{eq:ana.kappa_bound.g_is_positive}
    g(\eta_1, \ldots, \eta_{N-1}) \ge 0
  \]
  on $\Omega = \{\eta \in \Reals^{N - 1}\colon 0 \le \eta_i \le 1 \quad i = 1, \ldots, N - 1\}$.
  For this purpose, we use the \emph{Hopf Maximum Principle}~\cite{Taylor}:
  Since $g$ is linear in each $\eta_i$, we have
  \[
    \laplace g (\eta) = \sum_i \partial_{\eta_i}^2 g (\eta) = 0,
  \]
  $g$ and $\Omega$ fulfill the conditions of \cref{thm:ana.max_principle} on page \pageref{thm:ana.max_principle}.
  Therefore,
  \[
    \inf_{\eta \in \Omega} g(\eta) = \inf_{\eta \in \partial \Omega} g(\eta).
  \]
  Note that for $\eta \in \partial \Omega$, $\eta_j = 0$ or $\eta_j = 1$ for one $j$.
  Since $g$ is symmetric under permutation of the $\eta_i$ and under the transformation $\eta \mapsto 1 - \eta$, we can assume $\eta_1 = 1$ w.l.o.g.
  Therefore,
  \[
    \inf_{\eta \in \partial \Omega} g(\eta) = \inf_{{(0 \le \eta_i \le 1)}_{i > 1}} 1 - \prod_{i > 1} \eta_i = 0,
  \]
  and hence, \cref{eq:ana.kappa_bound.g_is_positive} holds.

  By substituting $\eta_i = \alpha_i^2$, \cref{eq:ana.kappa_bound.g_is_positive} gives
  \[
    \label{eq:ana.kappa_bound.sqrt_to_prod}
    \prod_i \left( 1 - \alpha_i^2 \right) \le 1 - \prod_i \alpha_i^2
  \]
  Therefore, we obtain the claim of this Lemma as follows
  \begin{align}
    C \prod_i \alpha_i - \prod_i \sqrt{1 - \alpha_i^2}
    &\ge C \sqrt{1 - \delta^2} - \delta \\
    &\ge C \left(1 - \delta\right) - \delta \\
  \end{align}
\end{proof}


%%%%%%%%%%%%%%%%%%%%%%%%%%%%%%%%%%%%%%%%%%%%%%%%%%%%%%%%%%%%%%%%%%%%%%%%%%%%%%%
\printbibliography[heading=bibintoc]


%%%%%%%%%%%%%%%%%%%%%%%%%%%%%%%%%%%%%%%%%%%%%%%%%%%%%%%%%%%%%%%%%%%%%%%%%%%%%%%
\chapter*{Acknowledgments}

%%%%%%%%%%%%%%%%%%%%%%%%%%%%%%%%%%%%%%%%%%%%%%%%%%%%%%%%%%%%%%%%%%%%%%%%%%%%%%%
\chapter*{Erklärung}

Ich versichere, dass ich die von mir vorgelegte Dissertation selbständig angefertigt, die benutzten Quellen und Hilfsmittel vollständig angegeben und die Stellen der Arbeit -- einschließlich Tabellen, Karten und Abbildungen --, die anderen Werken im Wortlaut oder dem Sinn nach entnommen sind, in jedem Einzelfall als Entlehnung kenntlich gemacht habe; dass diese Dissertation noch keiner anderen Fakultät oder Universität zur Prüfung vorgelegen hat; dass sie -- abgesehen von unten angegebenen Teilpublikationen -- noch nicht veröffentlicht worden ist, sowie, dass ich eine solche Veröffentlichung vor Abschluss des Promotionsverfahrens nicht vornehmen werde.
Die Bestimmungen der Promotionsordnung sind mir bekannt.
Die von mir vorgelegte Dissertation ist von Prof.\ Dr.\ David Gross betreut worden.

\vspace{4cm}

\hspace{2cm} Ort, Datum \hfill Unterschrift \hspace{2cm}

\vspace{\fill}
\subsection*{Teilpublikationen}
\begin{itemize}
  \item D.\ Suess, Ł.\ Rudnicki, T.\ O.\ Maciel, D.\ Gross: \textit{Error regions in quantum state tomography: computational complexity caused by geometry of quantum states}, New J.\ Phys.\ 19 093013 (2017)
  \item Ž.\ Stojanac, D.\ Suess, M.\ Kliesch: \textit{On the distribution of a product of N Gaussian random variables}, Proceedings Volume 10394, Wavelets and Sparsity XVII; 1039419 (2017)
  \item Ž.\ Stojanac, D.\ Suess, M.\ Kliesch, \textit{On products of Gaussian random variables}, arXiv:1711.10516
  \item D.\ Suess, M.\ Holzaepfel, \textit{mpnum: A matrix product representation library for Python}, Journal of Open Source Software, 2(20), 465 (2017)
\end{itemize}


\end{document}
