 % -*- root: ../thesis.tex -*-
\chapter{Tensors}
\label{chap:tensors}

In the last chapter, we discussed a solution to the phase retrieval problem based on ideas from \emph{low-rank matrix recovery}.
The latter can be summarized as follows.
Consider a matrix\footnote{%
  Here, we choose $X$ to be real to simplify some notation -- an adaption of the statements made to the complex case is straight forward.
}
$X \in \Reals^{d \times d}$ of rank $r$, i.e.\ $X$ has only $r$ non-zero singular values.
The goal is to reconstruct $X$ from $m$ linear measurements of the form
\[
  y_l = {\braket{A\ind{l}, X}}, \quad l=1,\ldots,m
  \label{eq:tensors.lin_measurements}
\]
where $\braket{A, B} = \tr (\transpose{A} B)$ denotes the Frobenius inner product, $A\ind{l} \in \Reals^{d \times d}$ the \emph{measurement matrices}, and $y_l \in \Reals$ the corresponding measurement data.
In other words, we are looking for a solution $X$ of the linear problem defined by \cref{eq:tensors.lin_measurements}.
Note that \cref{eq:tensors.lin_measurements} describes the idealized measurements in the absence of noise.

In case the $A_i$ span $\Reals^{d \times d}$, this can be solved simply by computing the (pseudo-)inverse similar to \cref{sub:ortho.linear_inversion}.
However, in the underdetermined case $m < d^{2}$, this constitutes an ill-posed linear problem.
Nevertheless, we can still single out a unique solution in some cases by exploiting the additional structure, namely the low-rank assumption.
For example, in \cref{thm:FIXME} we show that $m = \Order(d)$ random measurements sampled from an appropriate distribution are sufficient to reconstruct any positive semidefinite, Hermitian $X \in \Complex^{d \times d}$ with unit rank.
More generally, we can recover any $X \in \Reals^{d \times d}$ with $\rank X = r$ from $m = \Order(d r)$ randomly chosen measurements~\cite{Candes_2011_Tight,Kueng_2014_Low}.
Intuitively, this sample complexity is asymptotically optimal since we need at least $2 d r$ real parameters to specify the left- and right-singular vectors of $X$, see~\cite{Eldar_2012_Uniqueness,Li_2017_Optimal} for rigorous lower bounds.

But \cref{thm:FIXME} not only guarantees identifiability of $X$, it also provides a semidefinite program to compute the reconstruction.
\todo{Why is this solution?}
The existence of such an efficient algorithm is crucial since the straight-forward reconstruction via rank minimization
\[
  \estim{X} = \argmin_{X'} \rank X' \quad\mathrm{s.t.}\ \braket{A\ind{i}, X'} = y_i (i = 1, \ldots, m)
\]
is $\NP$-hard~\cite{Mesbahi}.\\

A closely related problem is studied under the name of \emph{compressed sensing}:
The goal is to recover an $s$-sparse vector $X \in \Reals^d$ from linear measurements as in \cref{eq:tensors.lin_measurements} with $\braket{\cdot, \cdot}$ denoting the Euclidean scalar product on $\Reals^d$.
Here, a vector is $s$-sparse if it has $s$ non-vanishing components with respect to a certain basis.
Therefore, the property of having low-rank may be regarded as a non-commutative analogue of sparsity since a matrix has low-rank if and only if it is sparse in its eigenbasis.
\todo[noline]{Check log factors}
Similar to the matrix case, $X$ can be recovered from $m = \Order(s \log d)$ randomly chosen measurement vectors $A\ind{l} \in \Reals^d$ using a linear program~\cite{Foucart_2013_Mathematical}.
Hence, the number of measurements required scales with the \emph{intrinsic complexity} of the sparse signal vector:
If we encode each component of $X$ using $N$ bits, we can encode the full vector using
\[
  N \times s + s \log d = \Order(s \log d)
\]
bits.
The second summand on the left hand side is necessary to encode the positions of the non-zero components.

In conclusion, both compressed sensing and low-rank matrix recovery provide techniques to efficiently reconstruct a low-complexity, compressible signal embedded in a higher-dimensional space from linear measurements.
Here, \quotes{efficiently} refers to both, low computational as well as sample complexity -- that is the number of measurements required scales with the intrinsic complexity of the signal and not the dimension of the ambient space.\\



In this chapter, we consider a natural extension of the ideas introduced above to the problem of recovering low-rank tensors.
For this purpose, consider a tensor $X \in \Reals^{d_1 \times \cdots \times d_N}$.
If not stated otherwise, we assume $d_1 = \cdots = d_N = d$ throughout this chapter.
Contrary to the case of matrices, there are many inequivalent definitions of \quotes{rank} for tensors.
The most natural generalization of the matrix rank is given by the \emph{canonical tensor rank} of $X$, which is defined by the smallest number of rank-1 tensors that add up to $X$~\cite{Kolda_2009_Tensor}
In other words, the canonical tensor rank is the smallest number $r$ such that
\[
  X = \sum_{l=1}^r a\ind{l}_1 \otimes \cdots a\ind{l}_N
  \label{eq:tensors.canonical_decomposition}
\]
with $a\ind{l}_i \in \Reals^{d_i}$.
The decomposition~\eqref{eq:tensors.canoncial_decomposition} is known under many names such as \emph{CANDECOMP/PARAFAC}~\cite{Kolda_2009_Tensor}.
Although it captures the natural structure of many problems well~\cite{FIXME} and only requires $\Order(r N d)$ parameters, it has shortcomings in practice:
First and foremost, computing the canonical tensor rank is $\NP$-complete~\cite{Hastad_1990_Tensor}.
Furthermore, the best rank-$r$ approximation of a given tensor may not exist~\cite{Kolda_2009_Tensor} and it is also hard to compute if it exists~\cite{Hillar_2013_Most}.
In general, the CANDECOMP/PARAFAC decomposition is hard to deal with computationally.
\todo{Some compressed sensing results for canonical low-rank tensors}

A different notion of tensor rank is induced by the tensor decomposition known under many different names such as \emph{matrix-product states} (MPS)~\cite{Garcia_2006_Matrix,Verstraete_2008_Matrix,Orus_2014_Practical} and \emph{finitely correlated states}~\cite{Fannes_1992_Finitely} in quantum physics or \emph{tensor train}~\cite{Oseledets_2011_TensorTrain} in the applied math community.
We introduce the exact definition in \cref{sec:tensors.mps}, but let use point out that a tensor $X \in \Reals^{d^{\otimes N}}$ with MPS-rank $r$ can be parametrized by $\Order{r^2 N d}$ real numbers.
This motivated the main question we are trying to answer in this chapter:
Can we efficiently reconstruct a tensor with low MPS-rank from few linear measurements?
In the ideal case, the number of measurements required for reconstructing a tensor with low MPS-rank would scale as $m = \Order{r^2 N d}$, which is an exponential improvement over the naïve linear inversion requiring $m = d^N$.
In contrast, compressed sensing or low-rank matrix recovery techniques can only yield polynomial improvement in the sample complexity.

This chapter is structured as follows:
In \cref{sec:tensors.mps}, we introduce the MPS tensor decomposition.
\Cref{sec:tensors.als} reports on work in progress, which is concerned with recovering tensors with low MPS-rank from few linear measurements using an \emph{Alternating Least Squares} (ALS) algorithm.
Finally, in \cref{sec:tensors.mpnum} we present the software library \mpnum dealing with tensors in MPS representation.
\mpnum was developed as part of this work to facilitate numerical computations in a user friendly and reusable manner.


\subsection*{Relevant publications}
\begin{itemize}
  \item Ž.\ Stojanac, D.\ Suess, M.\ Kliesch: \textit{On the distribution of a product of N Gaussian random variables}, Proceedings Volume 10394, Wavelets and Sparsity XVII; 1039419 (2017)
  \item Ž.\ Stojanac, D.\ Suess, M.\ Kliesch, \textit{On products of Gaussian random variables}, arXiv:1711.10516
  \item D.\ Suess, M.\ Holzaepfel, \textit{mpnum: A matrix product representation library for Python}, Journal of Open Source Software, 2(20), 465 (2017)
\end{itemize}
%%%%%%%%%%%%%%%%%%%%%%%%%%%%%%%%%%%%%%%%%%%%%%%%%%%%%%%%%%%%%%%%%%%%%%%%%%%%%%%%
\section{Matrix Product States}%
\label{sec:tensors.mps}

% graphical notation
% connected to the MPS/TT tensor decomposition; first use: Hidden Markov model -> factorization of probabliilty distributions. Quantum Physics -> Werner (finetly correlated states); tensor train -> machine learning
%   - here: MPS/TT rank: numerical advantageous, captures structure well of many problems in physics; especially well suited for linear local structure; recently quite popular
% low-rank tensor = low TT rank tensor

%%%%%%%%%%%%%%%%%%%%%%%%%%%%%%%%%%%%%%%%%%%%%%%%%%%%%%%%%%%%%%%%%%%%%%%%%%%%%%%%
\section{ALS for Tensors}%
\label{sec:tensors.als}

\begin{definition}
  HELLO
\end{definition}

%
% big problem: standard model (fully Gaussian measurements) also not efficient; partially solved in Zeljka

\subsection{Theory}%
\subsection{Numerics}%
% Implementation details; how to speed up initialization, why suitable for GPU%

%%%%%%%%%%%%%%%%%%%%%%%%%%%%%%%%%%%%%%%%%%%%%%%%%%%%%%%%%%%%%%%%%%%%%%%%%%%%%%%%
\section{The Python Library mpnum}%
\label{sec:tensors.mpnum}

