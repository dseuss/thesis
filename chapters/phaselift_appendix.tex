% -*- root: ../thesis.tex -*-
%%%%%%%%%%%%%%%%%%%%%%%%%%%%%%%%%%%%%%%%%%%%%%%%%%%%%%%%%%%%%%%%%%%%%%%%%%%%%%%%
\section{Experimental Details}%
\label{sec:pl.experimental_details}

\subsection{Reference Reconstructions}%
\label{sub:pl.hom_dip}

Since our goal is to benchmark the PhaseLift characterisation technique, and not the performance of the optical chip, we compare the experimental reconstructions to reconstructions obtained with a different technique.
These reference reconstructions are obtain in two steps.
First, we estimate the absolute value of each component from single photon data:
From \cref{eq:pl.intensities}, we see that by inserting single photons into the $k$-th input of the device -- that is choosing the standard basis vectors as inputs $\alpha =  e_k$ as inputs --  we can estimate $\abs{M_{i,k}}$.
For each input port, the counts of each detector are normalised to take into account the detector efficiencies and then divided by the total of the counts in all detectors.
The square roots of these numbers are used as the estimated amplitudes of the matrix elements.
Second, we estimate the phase of each component using HOM-dips~\cite{Hong_1987_Measurement}, following a similar approach to~\cite{Laing_2012_SuperStable,Dhand_2016_Accurate}.
However, this second step is time-consuming and only reliable for small devices.
Therefore, for the larger devices, we only perform the first step and compare only the magnitudes of the matrix elements of the reconstructions.

To perform the HOM experiments, the preparation-stage MZIs marked red in \cref{fig:experimental.schematic} were set to perform an identity transformation and both fibres carrying the photon pairs generated by the source were connected to input ports of the chip.
For each pairwise combination of the input modes of our matrices, we recorded the twofold coincidences among our detectors while changing the time delay of a photon relative to the other via a motorised translation stage.
The difference in coincidence counts as a function of delay were fit to the function:
\[
  f(\tau)=\left(1-c_1\exp[-((\tau - c_2)/c_3)^2]\mathrm{sinc}[(\tau - c_2)/c_6]\right)(c_4\tau + c_5 + c_7 \tau^2)
\]
where $\tau$ is the time delay of the photon and $\{c_i\}$ fit parameters.
The first term approximates the temporal envelope of a Gaussian photon subject to a top-hat filter and the second term adjusts for the decoupling resulting from the movement of the translation stage.
From the coefficient $c_1$ of this fit we retrieve a bare visibility of the two photon interference.
These bare visibilities are then rescaled by a reference visibility that accounts for the partial-distinguishability of the photons in their remaining degrees of freedom.
This was periodically assessed by implementing an effective 50:50 beam splitter on the chip and observing dip visibilities ranging from 0.965 and 0.985.

In order to retrieve the phase information of the matrix elements, the two photon interference fringes with the best estimates of the visibility were selected for each matrix to create a sequence that suffices to determine the phases of all the elements of the matrix.
Assuming that all the elements of the first row and of the first column are real and positive, the complex argument of the first phase factor retrieved associated to the element $M_{ij}$ is given by the following equation.
\[
\phi_{ij}=\pm  \arccos\left( - \frac{|M_{1,1}M_{i,j}|^2+|M_{1,j}M_{i,1}|^2}{2 |M_{1,1}M_{i,j}M_{1,j}M_{i,1}|} V_{1,j,1,i} \right)
\]
where $V_{1,j,1,i}$ is the visibility observed injecting photons into the ports 1 and j and detecting photons at the output ports 1 and i.
The sign $\pm$ in the equation is chosen to maximise the similarity to the target matrix.
Then, if we want to use a visibility $V_{\beta,j,\alpha,i}$ to deduce the phase $\phi_{ij}$ knowing $\phi_{\alpha \beta}$, $\phi_{\alpha j}$ and $\phi_{i \beta}$ we have
\[
\phi_{ij}=\pm   \arccos\left( - \frac{|M_{\alpha \beta}M_{i,j}|^2+|M_{\alpha,j}M_{i,\beta}|^2}{2 |M_{\alpha,\beta}M_{i,j}M_{\alpha,j}M_{i,\beta}|} V_{\beta,j,\alpha,i} \right) +\phi_{\alpha j} +\phi_{i \beta} -\phi_{\alpha \beta}
\]



\subsection{Data Analysis}%
\label{sub:pl.data_analysis}

\begin{table}
  \begin{tabular}{l | r r r}
    Dimension $n$ & 2 & 3 & 5 \\
    Gaussian & 20 & 30 & 40 \\
    RECR & 6 & 31 & 39 \\
  \end{tabular}
  \caption{%
    \label{tab:measurements}
    Total number of preparation vectors taken during experiment.
  }
\end{table}

As mentioned in the main text, we estimate the intensity measurements from single photon counting rates.
After correcting for detector efficiency, all counting rates are scaled by a constant such that the resulting intensities obey $\max_l \sum_j I_j(\alpha^{(l)}) = 1$.
This only amounts to scaling the transfer matrix by a constant, which does not influence the end result since we later rescale the obtained reconstruction appropriately (see \cref{eq:pl.experimental_details.data.scaling}).
However, this simple rescaling helps with numerical stability in the SDP solver.
We provide a ready for use implementation of the PhaseLift convex program~\eqref{eq:pl.PhaseLift} as well as related algorithms in the open source library \textsc{pypllon}~\cite{Suess_2017_Pypllon},

The post-processing of a reconstruction ${M}^\sharp$ consists of two steps:
First, we rescale the reconstruction by a constant such that
\[
  \label{eq:pl.experimental_details.data.scaling}
  \max_i \norm{(M^\sharp)_i}_{\ell_2} = 1,
\]
where $(M^\sharp)_i$ denotes the $i$-th row of $M^\sharp$.
In an ideal experiment, $M^\sharp$ would be unitary and, therefore, every row would have unit norm.
However, due to loss in the characterised circuit as well as detector inefficiencies, the norm of each row is smaller than one.
Since we cannot distinguish the two sources of loss in our current experimental setup, we cannot characterise the absolute photon loss in the circuit, but only the relative losses of the rows.
Estimating the dark counts in future experiments would enable characterising the absolute photon loss in the circuit as well.

The second post-processing step consists of fixing phases of the reconstructions:
Recall that we are only able to recover the transfer matrix up to its row phases since the global phases of the rows are lost in the intensity measurements.
Therefore, we fix the row phases of the PhaseLift reconstructions in \cref{fig:experimental.targetref} by minimizing the Frobenius distance to the target unitary and compute the error as
\[
  \min_{{\mu}: \abs{\mu_i} = 1}\left\|  {M}_\mathrm{target} -  \mathrm{diag} ({\mu}) {M}^\sharp) \right\|_2
\]
However, since the HOM-dip reconstruction is insensitive to global phases of the columns as well, we have to minimize both row and column phases for the HOM-dip reconstructions in \cref{fig:experimental.targetref}.
Furthermore, since in \cref{fig:experimental.overview} the HOM-dip reconstruction is taken as reference value, we have to minimize the row and column phases for all PhaseLift reconstructions in that picture as well.
The raw data as well as the analysis scripts are available at \url{https://github.com/dseuss/phaselift-paper}.
