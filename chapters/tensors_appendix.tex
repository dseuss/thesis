% -*- root: ../thesis.tex -*-
%%%%%%%%%%%%%%%%%%%%%%%%%%%%%%%%%%%%%%%%%%%%%%%%%%%%%%%%%%%%%%%%%%%%%%%%%%%%%%%%

\section{Tensors}%
\label{sec:tensors_appendix}


\subsection{Nonexistence of TRIP for rank-1 measurements}
\label{sub:tensors.trip}
\todo{proof why RIP not working for our measurements}


\subsection{Meijer G-functions}%
\label{sub:tensors.meijer}

Meijer G-functions are a family of special functions in one variable that is closed under several operations including
\[
  x \mapsto -x, \,
  x \mapsto 1/x,
  \text{ multiplication by } x^p,
  \text{ differentiation},
  \text{ and integration}.
\]
%
\begin{definition}\label{Def:MeijerG}
  For integers $m,n,p,q$ satisfying $0\leq m \leq q$, $0 \leq n \leq p$ and for $a_i, b_j \in \mathbb{C}$ (with $i=1,\ldots,p$; $j=1,\ldots,q$), the Meijer G-function
  $G_{p,q}^{m,n}\roundbra{\, \cdot \,\, \Big| \begin{matrix} a_1,a_2, \ldots a_p \\ b_1, b_2,\ldots, b_q\end{matrix}}$ is defined by the line integral
  \[
    \label{eq:the_G}
    G_{p,q}^{m,n}\roundbra{z \Big| \begin{matrix} a_1,a_2, \ldots a_p \\ b_1, b_2,\ldots, b_q\end{matrix}}
    =
    \frac{1}{2\pi \i} \int_{\mathcal{L}} \mathcal{H}_{p,q}^{m,n} (s) z^{-s} \, \d s \, ,
  \]
  with
  \[\label{Hmnpq}
    \mathcal{H}_{p,q}^{m,n}(s) :=   \frac{\prod_{j=1}^m \Gamma(b_j+s) \prod_{i=1}^n \Gamma(1-a_i-s)}{\prod_{i=n+1}^p \Gamma(a_i+s) \prod_{j=m+1}^q \Gamma(1-b_j-s)}
    \,.
  \]
  Here,
  \[
    z^{-s}=\exp\roundbra{-s \curlybra{\log |z|+ \i \arg z}}, \quad z \neq 0, \quad \i=\sqrt{-1},
  \]
  where $\log|z|$ represents the natural logarithm of $|z|$ and $\arg z$ is not necessarily the principal value.
  Empty products are identified with one.
  The parameter vectors $a$ and $b$ need to be chosen such that the poles
  \[\label{polesB}
    b_{j\ell} = -b_j - \ell \quad (j=1,2,\ldots,m;\, \ell=0,1,2,\ldots)
  \]
  of the gamma functions $s\mapsto \Gamma(b_j+s)$ and the poles
  \[\label{polesA}
    a_{ik} = 1-a_i+k \quad (i=1,2,\ldots,n;\, k=0,1,2,\ldots)
  \]
  of the gamma functions $s\mapsto \Gamma(1-a_i-s)$ do not coincide, i.e.
  \[
    b_j+\ell \neq a_i-k-1 \quad (i=1,\ldots,n;\, j=1,\ldots,m;\,  k,\ell=0,1,2,\ldots).
  \]
  The integral is taken over an infinite contour $\mathcal{L}$ that separates all  poles $b_{j\ell}$ in Eq.~\eqref{polesB} to the left and $a_{ik}$ in Eq.~\eqref{polesA} to the right of $\mathcal{L}$, and has one of the following forms:
  \begin{enumerate}
    \item $\mathcal{L}=\mathcal{L}_{-\infty}$ is a left loop situated in a horizontal strip starting at the point $-\infty+\i \phi_1$ and terminating at the point $-\infty + \i \phi_2$ with $-\infty < \phi_1 < \phi_2 < +\infty$;
    \item $\mathcal{L}=\mathcal{L}_{+\infty}$ is a right loop situated in a horizontal strip starting at the point $+\infty+\i \phi_1$ and terminating at the point $+\infty + \i \phi_2$ with $-\infty < \phi_1 < \phi_2 < +\infty$;
    \item $\mathcal{L}=\mathcal{L}_{\i \gamma \infty}$ is a contour starting at the point $\gamma-\i \infty$ and terminating at the point $\gamma+\i \infty$, where $\gamma \in \Reals$.
  \end{enumerate}
\end{definition}
