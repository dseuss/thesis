 % -*- root: ../thesis.tex -*-
%%%%%%%%%%%%%%%%%%%%%%%%%%%%%%

\section{Generalized Bloch Representation}
\label{sec:error.parametrisation}

Here, we provide the particular generalizations $\sigma_{i}$ of the Pauli matrices used in Sec.~\ref{sub:ortho.linear_inversion}.
These are exactly the generators of the group $\mathrm{SU}(d)$, see e.g.~\cite{Kimura_2003_Bloch,Byrd_2003_Characterization} for more details.
Denote by  ${\{\ket{i}\}}_i$ an orthonormal basis and let
\begin{align*}
  \Xi_{jk}^{(\textrm{Re})} &=  \ket{j}\bra{k} + \ket{k}\bra{j}, \\
  \Xi_{jk}^{(\textrm{Im})} &= -\ii \Big(\ket{j}\bra{k} - \ket{k}\bra{j}\Big), \\
  \Xi_{l}^{(\textrm{diag})} &= \sqrt{\frac{2}{l\left(l+1\right)}}\left(\sum_{j=1}^{l} \ket{j}\bra{j} - l \ket{l+1}\bra{l+1} \right).
\end{align*}
We now define the generalized Pauli matrices in terms of these auxiliary matrices:
\begin{align}
  \label{eq:parametrisation.x}
  \left\{ \sigma_{i}:i=1,\ldots,i_{d}\right\} &= \left\{ \Xi_{jk}^{(\textrm{Re})}:1\leq j<k\leq d\right\},  \\
  \label{eq:parametrisation.y}
  \left\{ \sigma_{i}:i=i_{d}+1,\ldots,2i_{d}\right\} &= \left\{ \Xi_{jk}^{(\textrm{Im})}:1\leq j<k\leq d\right\}, \\
  \label{eq:parametrisation.z}
  \left\{ \sigma_{i}:i=2i_{d}+1,\ldots,d^{2}-1\right\} &= \left\{ \Xi_{l}^{(\textrm{diag})}:1\leq l\leq d-1\right\},
\end{align}
where $i_{d}=d(d-1)/2$.
Note that the elements of the sets in Eq.~\eqref{eq:parametrisation.x}, \eqref{eq:parametrisation.y}, and \eqref{eq:parametrisation.z} generalize the Pauli matrices $\sigma_\mathrm{X}$, $\sigma_\mathrm{Y}$, and $\sigma_\mathrm{Z}$, respectively.
Since only this structure is crucial to our proof, the order of the elements in Eq.~\eqref{eq:parametrisation.x}--\eqref{eq:parametrisation.z} in arbitrary, and hence, the definition in terms of sets is well defined for our purposes.
