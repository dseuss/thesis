 % -*- root: ../thesis.tex -*-

\chapter{Conclusion}%
\label{chap:conclusion}

To summarize, we investigated three inference problem from quantum physics, which are subject to different types of constraints.
The main motivation of this work stems from the observation that exploiting additional structure in inference problems often helps to reduce their sample complexity.
However, the examples in this work differ in how taking the constraints into account affects their computational complexity.\\



The first inference problem under consideration is quantum state estimation -- reconstructing the density matrix of a quantum system from measurements.
More specifically, we are interested in optimal error regions that take the positive semi-definite constraint of physical states into account.
For this purpose, we show that deciding whether an ellipsoid is contained in the set of positive semi-definite matrices is $\NP$-hard.
As a consequence, computing the radius of optimal Bayesian credible ellipsoids for QSE is computationally intractable, whereas the unconstrained problem can be solved efficiently.
For Frequentist confidence regions, this result implies that computing any property of truncated confidence ellipsoids that is sensitive to truncation is hard as well.
In conclusion, although there are settings where taking into account the physical constraints of QSE drastically improves the power of the error region, doing so in an optimal way is computationally intractable.

Note that this work does not preclude the existence of algorithms for uncertainty quantification in QSE that work well-enough in practice.
Our hardness results relies on strong assumptions, some of which might be relaxed for practical applications.
For example, our results leave room for the existence of efficient approximate solutions.
Rather, our hardness result should be understood as an absolute upper bound on what such algorithms can achieve.\\


% furthemore, we propose a measruement ensemble for phase-retrieval tailored to applications in linerar optics
% in constrast to the Gaussian ensemble used in previous work, the RECR ensemble only requires preparation of 4 complex phases per mode and discrete magintudes -> able to calibrate experimetnal implemention better

In \cref{chap:phaselift}, we investigate characterizing linear optical circuits and the related phase retrieval problem.
To overcome to challenge of phase-insensitive measurements, we map the problem to rank-one matrix recovery.
By exploiting the exact rank-one constraint, we are able to perform reconstruction using an asymptotically optimal number of measurements.
Furthermore, our recovery protocol can be implemented efficiently using a positive-semidefinite program called \quotes{PhaseLift} and it is robust to noise as the rigorously proven recovery guarantees show.
From an experimentalist's point of view, characterization of linear-optical networks via PhaseLift is favourable because it reduces the number of different measurement configurations required.
As reconfiguring the chip for another input takes more time on the architecture used for the current experiment than the actual measuring process, the sample efficiency of our protocol reduces the total amount of time required.

We also propose a measurement ensemble for phase retrieval tailored to the application in optics.
In contrast to the Gaussian ensemble used in previous work, the RECR ensemble only necessitates the ability to prepare four complex phases per mode and discrete magnitudes.
\todo{Add phaselift conclusions}
This allows for calibrating the preparation stage more accurately, and hence, reduce the total error due to a mismatch of theoretical and implemented input vectors.\\


The problem of low-rank tensor reconstruction shows that in some cases constraints are necessary for an efficient solution.
High-order tensors are hard to deal with computationally due to the exponential scaling of the number of parameters.
This motivated the development of different tensor formats such as the MPS format considered here, which by construction reflects the correlation structure of certain tensors occurring in applications.
Therefore, they allow for an efficient representation of these relevant tensors.
Here, we answer the question whether such a tensor with efficient MPS representation can be reconstructed from few linear measurements.
In contrast to previous work, we are interested in both sample efficiency and computational complexity.
Hence, we consider product measurements, which are efficiently representable as well.

% finally demonstrate recovery of higher order tensors as well numerically
The analysis of the ALS algorithm yields a sufficient condition that guarantees successful recovery of any rank-one tensor using rank-one measurements.
As a prototypical example, we consider Gaussian product measurements, which are numerically shown to satisfy these conditions for a large variety of parameters.
Additionally, numerical reconstruction experiments show that we are able to reconstruct large tensors from serenely under sampled measurements.


The ALS algorithm using rank-one measurements combines both sample and computational efficiency for tensor reconstruction.
By exploiting the low-rank constraint we obtain an exponential improvement for the sampling rate compared to na\"ive approaches.
This reduction of the number of measurements necessary for recovery is also crucial for making the reconstruction computationally efficient.
The reduction of sample and computational complexity for low-rank tensor recovery is therefore qualitatively different from existing work on compressive sensing and low-rank matrix recovery.
For the latter, reconstruction without additional constraints is still feasible and, at least in the sense of polynomial scaling, still efficient.
