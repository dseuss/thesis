 % -*- root: ../thesis.tex -*-
\chapter{Uncertainty Quantification for quantum state estimation}
\label{chap:error}


%%%%%%%%%%%%%%%%%%%%%%%%%%%%%%%%%%%%%%%%%%%%%%%%%%%%%%%%%%%%%%%%%%%%%%%%%%%%%%%%
\section{Introduction to Statistics}
\label{sec:error.intro}

% two main flavours of statistics...
% task of parameter estimation, model
% parametric model: state space \Omega
However, even if our model describes the data perfectly we cannot exactly recover this value from a finite amount of data due to statistical fluctuations.
The concept of error bars, or more generally error regions, allows for quantifying the uncertainty of a given estimate. 

% uncertatiny quantification, problem with point estimators

\subsection{Frequentist Statistics}
\label{sub:intro.frequentist}

In the frequentist (or orthodox) framework, the probability of an outcome of a random experiment is defined in terms of its relative frequency of occurrence when the number of repetitions goes to infinity~\cite{Keynes_2007_Treatise,Kiefer_2012_Introduction}.
More precisely, denote the number of repetitions of an experiment by $T$ and the number of times the event under consideration $x$ occurred by $n_T$. 
Then, a Frequentist interprets the probability $\Prob(x)$ as the statement that if $T \to \infty$, $\frac{n_T}{T} \to \Prob(x)$.

% probabliities = hypothetical frequencies, not repeatable
For the task of parameter estimation, we assume that the observed data are generated from the parametric model with \quotes{true} parameter $\theta \in \Omega$, which is unknown.
From a finite number of observations $X_1, \ldots X_N$, we must construct an estimate for $\theta$ that is close to the true value in some sense.
The function $\hat\theta$ that maps observations to such an estimate is called a point estimator.
The quality of an estimator is measured by its risk function.
A risk function commonly used for continuous parameter spaces is the mean square error
\[
  \label{eq:frequentist.mean_square}
  \mathcal{L}_{\hat\theta}(\theta) := \Exp_\theta \left( \Norm{\theta - \hat\theta(X_1, \ldots, X_N)}^2 \right).
\]
Note that \cref{eq:frequentist.mean_square} -- and therefore the performance of a given estimator -- still depends on the unknown true value $\theta$.
Strategies to make statements independent of the true value include 
\begin{definition}
  \label{def:frequentist.optimality_conditions}
  \begin{itemize}
    \item $\hat\theta$ is called a \emph{uniformly best} estimator, if for all other estimators $\hat\theta'$ and all values of the true parameter $\theta \in \Omega$
    \[
      \mathcal{L}_{\hat\theta}(\theta) \le \mathcal{L}_{\hat\theta'}(\theta).
    \]

    \item $\hat\theta$ is called \emph{minimax}, if for all other estimators $\hat\theta'$
    \[
      \sup_{\theta\in\Omega} \mathcal{L}_{\hat\theta}(\theta) \le \sup_{\theta\in\Omega} \mathcal{L}_{\hat\theta'}(\theta).
    \]

    \item $\hat\theta$ is called \emph{best on average} w.r.t.\ a distribution of the true value $\theta \sim \Theta$ if for all other estimators $\hat\theta'$
    \[
      \Exp_{\theta \sim \Theta} \mathcal{L}_{\hat\theta}(\theta) \le  \Exp_{\theta \sim \Theta} \mathcal{L}_{\hat\theta'}(\theta).
    \]

    \item $\hat\theta$ is called \emph{admissible} if there is no other estimator $\hat\theta'$ such that
    \[
      \forall\theta \in \Omega\colon \mathcal{L}_{\hat\theta'}(\theta) \le \mathcal{L}_{\hat\theta}(\theta) 
      \quad\mbox{ and }\quad
      \exists\theta \in \Omega\colon \mathcal{L}_{\hat\theta'}(\theta) < \mathcal{L}_{\hat\theta}(\theta) 
    \]
  \end{itemize}
\end{definition}
\todo{Citation}
\todo{Discussion of different properties}
\todo{Choice is arbitrary!}
\todo{How does this connect with definition of probability?}


% examples (also depends on notion of volume), admissability
However, as already mentioned in the introduction, point estimators cannot convey uncertainty in the estimate. 
For this purpose we need to introduce a precise notion of \quotes{error bars}, namely \emph{confidence regions}.
A confidence region $\mathcal{C} \subset \Omega$ with coverage $\alpha \in [0,1]$ is a region estimator -- that is a function that maps observed data to a subset of the parameter space -- such that the true parameter is contained in $\mathcal{C}$ with probability greater than $\alpha$
\[
  \label{eq:frequentist.coverage}
  \forall \theta\in\Omega \colon \Prob_\theta\left( \mathcal{C}(X_1, \ldots, X_N) \ni \theta \right) \ge \alpha.
\]
Similar to point estimators, \cref{eq:frequentist.coverage} does not uniquely determine a confidence region construction.
Furthermore, additional constraints are necessary to exclude trivial constructions such as the following:
Take the region estimator, which is always equal to the full parameter space independent of the data $\mathcal{C}(X_1, \ldots X_N) = \Omega$, then 
\[
  \Prob_\theta\left(  \mathcal{C}(X_1, \ldots, X_N) \ni \theta  \right = 1 \ge \alpha
\]
for all confidence levels $\alpha$.
Although, this construction trivially fulfils the coverage condition~\eqref{eq:frequentist.coverage}, it does not provide useful information on the uncertainty as it does not restrict the parameter space at all.







\todo{Today: Asymptotic optimal, adaptive, ...}